% Copyright 2003--2007 by Till Tantau
% Copyright 2010 by Vedran Mileti\'c
%
% This file may be distributed and/or modified
%
% 1. under the LaTeX Project Public License and/or
% 2. under the GNU Free Documentation License.
%
% See the file doc/licenses/LICENSE for more details.

% $Header: /home/vedranm/bitbucket/beamer/doc/beamerug-nonpresentation.tex,v ca2315a97e66 2010/07/12 17:10:04 rivanvx $

\section{Creating Handouts and Lecture Notes}
\label{section-modes}

During a presentation it is very much desirable that the audience has a \emph{handout} or even \emph{lecture notes} available to it. A handout allows everyone in the audience to individually go back to things he or she has not understood.

Always provide handouts \emph{as early as possible}, preferably weeks before the talk. Do \emph{not} retain the handout till the end of the talk.

The \beamer\ package offers two different ways of creating special versions of your talk; they are discussed in the following. The first, easy, way is to create a handout version by adding the |handout| option, which will cause the document to be typeset in |handout| mode. It will ``look like'' a presentation, but it can be printed more easily (the overlays are ``flattened''). The second, more complicated and more powerful way is to create an independent ``article'' version of your presentation. This version coexists in your main file.


\subsection{Creating Handouts Using the Handout Mode}
\label{handout}

The easiest way of creating a handout for your audience (though not the most desirable one) is to use the |handout| option. This option works exactly like the |trans| option. An elaborated example of different overlay specifications for the presentation, the handout, and the transparencies can be found in the file |beamerexample1.tex|.

\begin{classoption}{handout}
  Create a version that uses the |handout| overlay specifications.

  You might wish to choose a different color and/or presentation theme for the handout.
\end{classoption}

When printing a handout created this way, you will typically wish to print at least two and possibly four slides on each page. The easiest way of doing so is presumably to use |pgfpages| as follows:

\begin{verbatim}
\usepackage{pgfpages}
\pgfpagesuselayout{2 on 1}[a4paper,border shrink=5mm]
\end{verbatim}

Instead of |2 on 1| you can use |4 on 1| (but then you have to add |landscape| to the list of options) and you can use, say, |letterpaper| instead of |a4paper|.


\subsection{Creating Handouts Using the Article Mode}
\label{section-article}

In the following, the ``article version'' of your presentation refers to a normal \TeX\ text typeset using, for example, the document class |article| or perhaps |llncs| or a similar document class. This version of the presentation will typically follow different typesetting rules and may even have a different structure. Nevertheless, you may wish to have this version coexist with your presentation in one file and you may wish to share some part of it (like a figure or a formula) with your presentation.

In general, the article version of a talk is better suited as a handout than a handout created using the simple |handout| mode since it is more economic and can include more in-depth information.

\subsubsection{Starting the Article Mode}

The article mode of a presentation is created by specifying |article| or |book| or some other class as the document class instead of |beamer| and by then loading the package |beamerarticle|.

The package |beamerarticle| defines virtually all of \beamer's commands in a way that is sensible for the |article| mode. Also, overlay specifications can be given to commands like |\textbf| or |\item| once |beamerarticle| has been loaded. Note that, except for |\item|, these overlay specifications also work: by writing |\section<presentation>{Name}| you will suppress this section command in the article version. For the exact effects overlay specifications have in |article| mode, please see the descriptions of the commands to which you wish to apply them.

\begin{package}{{beamerarticle}\opt{|[|\meta{options}|]|}}
  Makes most \beamer\ commands available for another document class.

  The following \meta{options} may be given:
  \begin{itemize}
  \item
    \declare{|activeospeccharacters|} will leave the character code of the pointed brackets as specified by other packages. Normally, \beamer\ will turn off the special behavior of the two characters |<| and |>|. Using this option, you can reinstall the original behavior at the price of possible problems when using overlay specifications in the |article| mode.
  \item
    \declare{|noamsthm|} will suppress the loading of the |amsthm| package. No theorems will be defined.
  \item
    \declare{|notheorem|} will suppress the definition of standard environments like |theorem|, but |amsthm| is still loaded and the |\newtheorem| command still makes the defined environments overlay-specification-aware. Using this option allows you to define the standard environments in whatever way you like while retaining the power of the extensions to |amsthm|.
  \item
    \declare{|envcountsect|} causes theorem, definitions and the like to be numbered with each section. Thus instead of Theorem~1 you get Theorem~1.1. We recommend using this option.
  \item
    \declare{|noxcolor|} will suppress the loading of the |xcolor| package. No colors will be defined.
  \end{itemize}

  \example
\begin{verbatim}
\documentclass{article}
\usepackage{beamerarticle}
\begin{document}
\begin{frame}
  \frametitle{A frame title}
  \begin{itemize}
\item<1-> You can use overlay specifications.
\item<2-> This is useful.
  \end{itemize}
\end{frame}
\end{document}
\end{verbatim}
\end{package}

There is one remaining problem: While the |article| version can easily \TeX\ the whole file, even in the presence of commands like |\frame<2>|, we do not want the special article text to be inserted into our original \beamer\ presentation. That means, we would like all text \emph{between} frames to be suppressed. More precisely, we want all text except for commands like |\section| and so on to be suppressed. This behavior can be enforced by specifying the option |ignorenonframetext| in the presentation version. The option will insert a |\mode*| at the beginning of your presentation.

The following example shows a simple usage of the |article| mode:

\begin{verbatim}
\documentclass[a4paper]{article}
\usepackage{beamerarticle}
%%\documentclass[ignorenonframetext,red]{beamer}

\mode<article>{\usepackage{fullpage}}
\mode<presentation>{\usetheme{Berlin}}

%% everyone:
\usepackage[english]{babel}
\usepackage{pgf}

\pgfdeclareimage[height=1cm]{myimage}{filename}

\begin{document}

\section{Introduction}

This is the introduction text. This text is not shown in the
presentation, but will be part of the article.

\begin{frame}
  \begin{figure}
    % In the article, this is a floating figure,
    % In the presentation, this figure is shown in the first frame
    \pgfuseimage{myimage}
  \end{figure}
\end{frame}

This text is once more not shown in the presentation.

\section{Main Part}

While this text is not shown in the presentation, the section command
also applies to the presentation.

We can add a subsection that is only part of the article like this:

\subsection<article>{Article-Only Section}

With some more text.

\begin{frame}
  This text is part both of the article and of the presentation.
  \begin{itemize}
\item This stuff is also shown in both version.
\item This too.
  \only<article>{\item This particular item is only part
      of the article version.}
\item<presentation:only@0> This text is also only part of the article.
  \end{itemize}
\end{frame}
\end{document}
\end{verbatim}

There is one command whose behavior is a bit special in |article| mode: The line break command |\\|. Inside frames, this command has no effect in |article| mode, except if an overlay specification is present. Then it has the normal effect dictated by the specification. The reason for this behavior is that you will typically inserts lots of |\\| commands in a presentation in order to get control over all line breaks. These line breaks are mostly superfluous in |article| mode. If you really want a line break to apply in all versions, say |\\<all>|. Note that the command |\\| is often redefined by certain environments, so it may not always be overlay-specification-aware. In such a case you have to write something like |\only<presentation>{\\}|.

\subsubsection{Workflow}
\label{section-article-version-workflow}

The following workflow steps are optional, but they can simplify the creation of the article version.

\begin{itemize}
\item
  In the main file |main.tex|, delete the first line, which sets the document class.
\item
  Create a file named, say, |main.beamer.tex| with the following content:

\begin{verbatim}
\documentclass[ignorenonframetext]{beamer}
% $Header: /home/vedranm/bitbucket/beamer/solutions/conference-talks/conference-ornate-20min.en.tex,v 90e850259b8b 2007/01/28 20:48:30 tantau $

\documentclass[portuges]{beamer}

\mode<presentation>
{
  \usetheme{Warsaw}
  \setbeamercovered{transparent}
}

\usepackage{babel}

\usepackage[utf8x]{inputenc}

\usepackage{times}
\usepackage[T1]{fontenc}
\usepackage{amsmath}
\usepackage{latexsym}
%\usepackage{amstex}

\title[Fuzzy \& SIAD] %
{Lógica Fuzzy aplicada a \\Sistemas de Informação de Apoio à Decisão}


\author[António Sérgio Matos da Silva] %
{António Sérgio Matos da Silva \\ \texttt{an.silva@logica.com}}

\institute[Logica] % 
{
  Telecommunication Business\\
  Logica
}

\date[Reunião Mensal Zon] % 
{\today\\Reunião Mensal Zon}

\subject{Theoretical Computer Science}


% Delete this, if you do not want the table of contents to pop up at
% the beginning of each subsection:
\AtBeginSubsection[]
{
  \begin{frame}<beamer>{Programa}
    \tableofcontents[currentsection,currentsubsection]
  \end{frame}
}

\pgfdeclareimage[height=0.5cm]{logo}{img/logo}
\logo{\pgfuseimage{logo}}

%\beamerdefaultoverlayspecification{<+->}

%%%%%%%%%%%%%%%%%%%%%%%%%%%%%%%%%%%%MY SPECS
\newtheorem{convention}[theorem]{Convenção}
\newtheorem{proposition}[theorem]{Proposição}

\begin{document}

\begin{frame}
  \titlepage
\end{frame}

\begin{frame}
\begin{quote}
``À medida que a complexidade aumenta, 
as declarações precisas perdem relevância e as declarações relevantes perdem precisão.'' 
\end{quote}
\begin{quote}
Lofti Zadeh
\end{quote}
\end{frame}

\begin{frame}{Programa}
  \tableofcontents[pausesections]
\end{frame}
%%%%%%%%%%%%%%%%%%%%%%%%%%%%%%APRESENTACAO%%%%%%%%%%%%%%%%%%%%%%%%%%%%%%%%

\section{Breve contextualização Teórica}
\subsection{Motivação}

\begin{frame}{Como ``Raciocina'' um Computador Tradicional?}
\pgfdeclareimage[width=.6\textwidth]{pcthink}{img/pcthink}
\begin{figure}
\centering
\pgfuseimage{pcthink}
\caption{Funcionamento de um PC}
\end{figure}
\end{frame}


\begin{frame}{Pensamento Lógico Humano}
\pgfdeclareimage[width=.6\textwidth]{larrythink}{img/larrythink}
\begin{figure}
\centering
\pgfuseimage{larrythink}
\caption{Funcionamento do cérebro de Larry Wall}
\end{figure}
\end{frame}

\begin{frame}{Motivação}
\begin{itemize}[<+->]
	\item Tradicionalmente, o comportamento dos sistemas é realizado
		pela representação da lógica de Boole, aceitando apenas dois resultados
		\alert{zero ou um}, \alert{verdadeiro ou falso}, tudo ou nada.
	\item No entanto, em muitas situações não relevamos esta dicotomia comportamental.
		\begin{itemize}
			\item Apesar de à partida, uma pessoa com $1.75\ m$ ser alta, esta não é assim
			tão alta;
			\item Quando alguém nos diz que nos ama, \alert{não sabemos o quanto} esta nos ama;
			\item Nem sempre precisamos de obter um resultado baseado numa \alert{certeza},
			ficamos satisfeitos apenas com um certo grau de \alert{confiança}.
		\end{itemize}		 
\end{itemize}

\end{frame}

\begin{frame}{Analogia}
	\pgfdeclareimage[height=2.5cm]{rightcurve}{img/rightcurve}
	\begin{example}
			\begin{figure}
			\centering
			\pgfuseimage{rightcurve}
			\end{figure}

			Pretende-se comparar os dois mecanismos
			para efectuar a curva à direita.		 	
	\end{example}
\end{frame}
\begin{frame}{Analogia}
\begin{columns}[T]
  		\begin{column}{5cm}
		    \begin{block}{Booleana}<1->  				
				\begin{enumerate}[<+->]
					\item Pressionar o travão com uma força de $20\ Newtons$.
					\item Inclinar o volante $15\ graus$ para a Direita
					\item Colocar o volante na posição inicial ($0\ graus$).
				\end{enumerate}		
			\end{block}
		  \end{column}
		  \begin{column}{5cm}
		    \begin{block}{Humana}<4->
  			\begin{enumerate}[<+->]
					\item Reduza a velocidade.

					\item Vire \alert{um pouco} para a direita
					\item Vire \alert{mais um pouco} para a direita
					\item Siga em frente.
				\end{enumerate}		
		    \end{block}

		  \end{column}
		\end{columns}

\end{frame}


\subsection{História e Uso da Lógica Fuzzy}

\begin{frame}{Resumo Histórico\ldots}

	\pgfdeclareimage[height=3cm]{lukasiewicz}{img/Lukasiewicz}
   \begin{columns}
	\begin{column}{7cm}
	\begin{itemize}[<+->]
		\item[1930] Jan Lukasiewz propôs o estudo de termos qualitativos do tipo alto, velho e quente 
		e propôs a idéia da utilização de um intervalo de valores entre $0$ e $1$ para descrever a veracidade
		de uma dada afirmação;
		\item[1937] Max Black definiu o primeiro conjunto fuzzy e descreveu algumas idéias básicas de operações com estes.
	\end{itemize}
	\end{column}
	\begin{column}{4cm}
		\begin{figure}
			\centering
			\pgfuseimage{lukasiewicz}
			\caption{Jan Lukasiewicz (1878--1956)}
		\end{figure}
	\end{column}	
   \end{columns}
\end{frame}

\begin{frame}{Resumo Histórico\ldots}
\pgfdeclareimage[height=3cm]{zadeh}{img/zadeh}
   \begin{columns}
	\begin{column}{7cm}
	\begin{itemize}[<+->]
		\item[1965] Lofti Zadeh publicou o artigo Fuzzy Sets que ficou
		conhecido como a origem da Lógica Fuzzy. Zadeh é conhecido como
		o ``mestre'' da Lógica Fuzzy.
	\end{itemize}
	\end{column}
	\begin{column}{4cm}
		\begin{figure}
			\centering
			\pgfuseimage{zadeh}
			\caption{Lofti Zadeh (1921--Actualidade)}
		\end{figure}
	\end{column}	
   \end{columns}
\end{frame}

\begin{frame}{Uso da Lógica Fuzzy}
\begin{itemize}
	\item[1970] Primeira aplicação da Lógica Fuzzy na engenharia de controlo;
	\item[1975] Introdução da Lógica Fuzzy no Japão;
        \item[1985] Ampla utilização no Japão;
        \item[1990] Ampla utilização na Europa;
        \item[1995] Ampla utilização nos EUA;
        \item[1996] 1100 aplicações com Lógica Fuzzy publicadas; 
	\item[2000] Aplicada a finanças e controle multi-variável.
\end{itemize}
\end{frame}

\begin{frame}{Porquê usar Fuzzy?}
\begin{block}{Vantagens da Lógica Fuzzy}
\begin{itemize}
	\item<1-> Robusta porque não requer entradas precisas;
	\item<2-> Facilmente modificável pois é baseada em regras;
	\item<3-> Evita o formalismo matemático para sistemas não lineares;
	\item<3-> Solução rápida e barata para sistemas complexos não lineares;
	\item<4-> Implementável em microprocessadores.
\end{itemize}
\end{block}
\end{frame}


\subsection{Fundamentos da Lógica Fuzzy}

\begin{frame}{Conceito de Difusidade}
	\pgfdeclareimage[height=3cm]{arcond}{img/arcond}  	
	\begin{figure}
			\centering
			\pgfuseimage{arcond}
	\end{figure}
		
	\begin{example}[O caso do ar condicionado]
  		No edifício Pinta o ar condicionado encontra-se constantemente
		avariado. Pretende-se desenvolver uma lógica que pretende 
		avaliar se uma dada temperatura é \alert{confortável} ou não.
	\end{example}
\end{frame}

\begin{frame}{Conceito de Difusidade}
	\pgfdeclareimage[width=.6\textwidth]{fuzzy1}{img/fuzzy1}	
	\begin{quote}
		Numa primeira análise vem o José e diz que a temperatura ideal
		para ele é de exactamente $20º\ C$.
	\end{quote}
	\begin{figure}
			\centering
			\pgfuseimage{fuzzy1}
			\caption{Função de Verdade de uma Lógica tipicamente Booleana.}
	\end{figure}
\end{frame}

\begin{frame}{Conceito de Difusidade}
	\pgfdeclareimage[width=.5\textwidth]{fuzzy2}{img/fuzzy2}	
	\begin{quote}
		No entanto o Miguel discorda com valor estipulado anteriormente. ``Vamos
		lá ver! Se a temperatura for 18 ou 22 não deixa de ser bom também!'' 
	\end{quote}
	\begin{figure}
			\centering
			\pgfuseimage{fuzzy2}
			\caption{Função de Verdade de uma Lógica na Cabeça do ``Miguel''.}
	\end{figure}
\end{frame}

\begin{frame}{Conceito de Difusidade}
	\pgfdeclareimage[width=.4\textwidth]{fuzzy3}{img/fuzzy3}	
	\begin{quote}
		No dia seguinte já com o ar condicionado a funcionar e uma
		temperatura a funcionar o Miguel pergunta ao Pedro.\\
		``Bom dia! Está bom tempo hoje!''\\
		Ao que este responde:\\
		``Sim está. \alert{Nem muito quente nem muito frio}\ldots''  
	\end{quote}
	\begin{figure}
			\centering
			\pgfuseimage{fuzzy3}
			\caption{Função de Verdade de uma Lógica Fuzzy.}
	\end{figure}
\end{frame}

\begin{frame}{Conceito de Difusidade}
	\pgfdeclareimage[width=.6\textwidth]{fuzzy3_1}{img/fuzzy3_1}	
	\begin{quote}
		Na verdade o que o Pedro pensou\ldots
	\end{quote}
	\begin{figure}
			\centering
			\pgfuseimage{fuzzy3_1}
			\caption{Função de Verdade de uma Lógica Fuzzy representando mais do que uma valoração ao mesmo tempo.}
	\end{figure}
\end{frame}

\begin{frame}{Definições Fuzzy --- Etiqueta linguística}
	\begin{definition}
		A descrição de inúmeras situações concretas referentes a uma dada variável faz-se por intermédio de \alert{etiquetas
		linguísticas}. Estas representam o carácter \alert{qualitativo} de todas as possibilidades de uma dada variável.
	\end{definition}
	
	\begin{exampleblock}{Etiquetas Bivalentes}<2->
		\begin{itemize}
			\item Lento;
			\item Rápido.
		\end{itemize}	
	\end{exampleblock}
\end{frame}

\begin{frame}{Definições Fuzzy --- Exemplo de Etiquetas Linguísticas}
	\begin{exampleblock}{Etiquetas Trivalentes}
		\begin{itemize}
			\item Baixo;
			\item Médio;
			\item Alto.
		\end{itemize}	
	\end{exampleblock}
	
	\begin{exampleblock}{Etiquetas Multivalentes}
		\begin{itemize}
			\item Muito Frio
			\item Frio;
			\item Moderado;
			\item Quente;
			\item Muito Quente.
		\end{itemize}	
	\end{exampleblock}
\end{frame}

\begin{frame}{Definições Fuzzy --- Variável Linguística}
		\pgfdeclareimage[width=.6\textwidth]{continuousfuzzy}{img/fuzzy3}	
	\begin{definition}<1->
		Uma variável linguística pode ser contínua ou discreta.
		\end{definition}
	\begin{example}<2->[Temperatura]	
	\begin{figure}
			\centering
			\pgfuseimage{continuousfuzzy}
			\caption{Exemplo de Variável Linguística contínua.}
	\end{figure}
	\end{example}
\end{frame}

\begin{frame}
	\pgfdeclareimage[width=.6\textwidth]{discretefuzzy}{img/fuzzydiscrete}
	\begin{example}[Rodas de um camião]
	\begin{figure}
			\centering
			\pgfuseimage{discretefuzzy}
			\caption{Exemplo de Variável Linguística Discreta.}
	\end{figure}	
	\end{example}
\end{frame}

\begin{frame}{Definições Fuzzy --- Conjuntos Fuzzy no Mundo Real}
	\pgfdeclareimage[width=.4\textwidth]{realworldfuzzy}{img/realworldfuzzy}
	\begin{convention}
		Geralmente os Sistemas de Informação baseados num sistema de decisão Fuzzy
		usam a seguinte simplificação das etiquetas. A partir de agora usaremos também
		esta notação.
	\end{convention}

	\begin{example}[Conjunto Fuzzy no Mundo Real]
	\begin{figure}
			\centering
			\pgfuseimage{realworldfuzzy}
			\caption{Exemplo de Variável Linguística Contínua do Mundo Real.}
	\end{figure}	
	\end{example}
\end{frame}

\frame[label=eqpertenca]
{
  \frametitle{Definições Fuzzy --- Etiquetas e Conjuntos Fuzzy}
	\pgfdeclareimage[width=.4\textwidth]{pertenca}{img/pertenca}  	
	\begin{definition}
		Seja $X$ uma Etiqueta Linguística. Esta é representada por um Conjunto Fuzzy $\mathcal{C}$ descrito pela
		função de pertença $\mu_X(x_0)$.
	\end{definition}
		
	\begin{figure}
			\centering
			\pgfuseimage{pertenca}
			\caption{Etiqueta Linguística e respectiva Função Pertença.}
	\end{figure}		

    \hyperlink{eqpertencadetail}{\beamergotobutton{Equação Detalhe}}
}

\begin{frame}{Definições Fuzzy --- Suporte}
\pgfdeclareimage[width=.4\textwidth]{support}{img/support}  	
\begin{definition}
		Seja $X$ uma Etiqueta Linguística. Designa-se por Suporte de $X$ ($\mathcal{S}_X$)
		a zona em que a sua função pertença não é nula. Ou seja: $\{\mu_X(x_0) \ \vert \ \mu_X(x_0) \neq 0 \land x_0 \in [0,1] \}$
	\end{definition}

	\begin{figure}
			\centering
			\pgfuseimage{support}
			\caption{Suporte de uma Etiqueta Linguística}
	\end{figure}
\end{frame}

\begin{frame}{Definições Fuzzy --- Núcleo}
\pgfdeclareimage[width=.4\textwidth]{nucleo}{img/nucleo}  	
\begin{definition}
		Seja $X$ uma Etiqueta Linguística. Designa-se por Núcleo de $X$ ($\mathcal{N}_X$)
		a zona em que a sua função pertença é máxima. Ou seja: $\{\mu_X(x_0) \ \vert \ \mu_X(x_0) = 1 \land x_0 \in [0,1] \}$
	\end{definition}

	\begin{figure}
			\centering
			\pgfuseimage{nucleo}
			\caption{Núcleo de uma Etiqueta Linguística}
	\end{figure}
\end{frame}

\begin{frame}{Definições Fuzzy --- Variável Linguística Normada}  	
\pgfdeclareimage[width=.5\textwidth]{notnormada}{img/notnormada}  	
\begin{definition}
		Uma Variável Línguística diz-se Normada se esta gera um discurso limitado pelo 
		conjunto $[-1,1]$
	\end{definition}
		\begin{figure}
			\centering
			\pgfuseimage{notnormada}
			\caption{Variável não normada.}
	\end{figure}
\end{frame}

\begin{frame}{Definições Fuzzy --- Variável Linguística Normada Exemplo}  	
\pgfdeclareimage[width=.6\textwidth]{normada}{img/normada}  	
	\begin{example}[Variável Linguística Normada]		
	\begin{figure}
			\centering
			\pgfuseimage{normada}
			\caption{Variável normada à escala 1/120.}
	\end{figure}
	\end{example}
\end{frame}

\begin{frame}{Operações Com Conjuntos Fuzzy}
	\begin{proposition}
		Seja $\mho$ o Universo de uma Variável Linguística
		e sejam $\mathcal{A}$ e $\mathcal{B}$ dois Conjuntos Fuzzy 
		definidos por:
		$$
			\begin{array}{ccc}
				\mathcal{A} & = & \{ (x,\mu_\mathcal{A}(x))\ \vert\ x \in \mho \land \mu_\mathcal{A}(x) \in [0,1] \} \\
				\mathcal{B} & = & \{ (x,\mu_\mathcal{B}(x))\ \vert\ x \in \mho \land \mu_\mathcal{B}(x) \in [0,1] \}
			\end{array}
		$$   
	\end{proposition}
\end{frame}

\begin{frame}{Operações Com Conjuntos Fuzzy --- União}
	\pgfdeclareimage[width=.4\textwidth]{orvenn}{img/orvenn}  	

	\begin{definition}[União]
	A união entre $\mathcal{A}$ e $\mathcal{B}$ é definida por:
	$$	
	\mathcal{A} \cup \mathcal{B} = \{ ( x , max(\mu_\mathcal{A}(x),\mu_\mathcal{B}(x)))\ \vert\ x \in \mho \} 
	$$
	\end{definition}	
\begin{figure}
			\centering
			\pgfuseimage{orvenn}
			\caption{Diagrama de Venn da União de dois Conjuntos}
	\end{figure}	
\end{frame}

\begin{frame}{Operações Com Conjuntos Fuzzy --- Intersecção}
	\pgfdeclareimage[width=.4\textwidth]{andvenn}{img/andvenn}  	

	\begin{definition}[Intersecção]
	A Intersecção entre $\mathcal{A}$ e $\mathcal{B}$ é definida por:
	$$	
	\mathcal{A} \cap \mathcal{B} = \{ ( x , min(\mu_\mathcal{A}(x),\mu_\mathcal{B}(x)))\ \vert\ x \in \mho \} 
	$$
	\end{definition}	
\begin{figure}
			\centering
			\pgfuseimage{andvenn}
			\caption{Diagrama de Venn da Intersecção de dois Conjuntos}
	\end{figure}	
\end{frame}

\begin{frame}{Operações Com Conjuntos Fuzzy --- Complemento}
	\pgfdeclareimage[width=.25\textwidth]{notvenn}{img/notvenn}  	

	\begin{definition}[Complemento]
	O Complemento de $\mathcal{A}$ é definido por:
	$$	
	\lnot\mathcal{A} = \{ ( x , \mu_{\lnot\mathcal{A}}(x) )\ \vert\ x \in \mho \land \mu_{\lnot\mathcal{A}}(x) = 1 - \mu_{\mathcal{A}}(x) \} 
	$$
	\end{definition}	
\begin{figure}
			\centering
			\pgfuseimage{notvenn}
			\caption{Diagrama de Venn do Complemento de um Conjunto}
	\end{figure}	
\end{frame}

\begin{frame}{Modificadores Linguísticos}
	\begin{definition}
		Seja $\mathcal{A}$ um conjunto Fuzzy intervalar com a função de pertinência $\mu_\mathcal{A}{x}$.
		Então o \emph{Modificador Línguístico} de $\mathcal{A}$ é uma função intervalar $\mathcal{M}$ definida por:
		$$\mathcal{M}:\mathcal{I}[0,1] \to \mathcal{I}[0,1] $$
		que age na função pertinência $\mu_{\mathcal{I}\mathcal{A}}{x}$ transformando-a em $\mu_{m\mathcal{I}\mathcal{A}}{x}$ onde:
		$$\mu_{m\mathcal{I}\mathcal{A}}(x) = \mathcal{M}(\mu_{\mathcal{I}\mathcal{A}}(x))$$
	\end{definition}
\end{frame}

\begin{frame}{Modificadores Linguísticos --- Very}
	\pgfdeclareimage[height=3cm]{muito}{img/muito}  		
	\begin{definition}[Very]
		O modificador \alert{Very} (muito) define-se por:
		$$\mu_{v\mathcal{A}}{x} = \mu_{v\mathcal{A}}^2{x}$$ 
	\end{definition}
	\begin{figure}
			\centering
			\pgfuseimage{muito}
			\caption{Representação Gráfica do Modificador Very}
	\end{figure}	
\end{frame}

\begin{frame}{Modificadores Linguísticos --- Somewhat}
	\pgfdeclareimage[height=3cm]{pouco}{img/pouco}  		
	\begin{definition}[Somewhat]
		O modificador \alert{Somewhat} (pouco) define-se por:
		$$\mu_{v\mathcal{A}}{x} = \sqrt[2]{\mu_{v\mathcal{A}}{x}}$$ 
	\end{definition}
	\begin{figure}
			\centering
			\pgfuseimage{pouco}
			\caption{Representação Gráfica do Modificador Somewhat}
	\end{figure}	
\end{frame}

\begin{frame}{Modificadores Linguísticos --- Above}
	\pgfdeclareimage[height=3cm]{acima}{img/acima}  		
	\begin{definition}[Above]
		O modificador \alert{Above} (acima) define-se por:
		$$\mu_{v\mathcal{A}}{x} =\mu_{v\mathcal{A}}{x} - \delta$$ 
	\end{definition}
	\begin{figure}
			\centering
			\pgfuseimage{acima}
			\caption{Representação Gráfica do Modificador Above}
	\end{figure}	
\end{frame}

\begin{frame}{Modificadores Linguísticos --- Below}
	\pgfdeclareimage[height=3cm]{abaixo}{img/abaixo}  		
	\begin{definition}[Below]
		O modificador \alert{Below} (abaixo) define-se por:
		$$\mu_{v\mathcal{A}}{x} = \mu_{v\mathcal{A}}{x} + \delta$$ 
	\end{definition}
	\begin{figure}
			\centering
			\pgfuseimage{abaixo}
			\caption{Representação Gráfica do Modificador Below}
	\end{figure}	
\end{frame}

\begin{frame}{Modificadores Linguísticos --- Not}
	\pgfdeclareimage[height=3cm]{nao}{img/nao}  		
	\begin{definition}[Not]
		O modificador \alert{Not} (não) define-se por:
		$$\mu_{v\mathcal{A}}{x} = 1 - \mu_{v\mathcal{A}}{x}$$ 
	\end{definition}
	\begin{figure}
			\centering
			\pgfuseimage{nao}
			\caption{Representação Gráfica do Modificador Not}
	\end{figure}	
\end{frame}

\begin{frame}{Modificadores Linguísticos --- Not Very}
	\pgfdeclareimage[height=3cm]{naomuito}{img/naomuito}  		
	\begin{definition}[Not Very]
		O modificador \alert{Not Very} (não muito) define-se por:
		$$\mu_{v\mathcal{A}}{x} = 1 - \mu_{v\mathcal{A}}^2{x}$$ 
	\end{definition}
	\begin{figure}
			\centering
			\pgfuseimage{naomuito}
			\caption{Representação Gráfica do Modificador Not Very}
	\end{figure}	
\end{frame}

\begin{frame}{Inferência}
\begin{block}{Interferência}
\begin{itemize}[<+->]
\item
Para fazer deduções com conjuntos difusos utilizam-se \alert{regras de inferência},
		formatando afirmações condicionais como implicações do tipo ``if-then'';
\item O antecedente
		diz respeito às ``condições lógicas'' impostas sobre essa variável linguística;
\item O consequente
		diz respeito às ``acções'' decorrentes dessas condições na variável de saída;
\item No controlo difuso costumam haver múltiplas regras de inferência, de acordo com a natureza 
dos estados medidos no processo.
\end{itemize}
\end{block}
\end{frame}

\begin{frame}{Inferência}
\begin{example}[Base de Regras]
$$
	\begin{array}{ccccc}
	if & (antecedente_1) & then & (consequente_1) & or \\
	if & (antecedente_2) & then & (consequente_2) & or \\
	 &  & ... &  & \\
	if & (antecedente_n) & then & (consequente_n) &
	\end{array}
$$
\end{example}
\end{frame}

\begin{frame}{Pausa\ldots}
\pgfdeclareimage[width=0.9\textwidth]{kitkat}{img/kit_kat}  	
\begin{figure}
			\centering
			\pgfuseimage{kitkat}
	\end{figure}		
\end{frame}

\section{Estado da Arte}
\subsection{Estilo Mamdami}

\begin{frame}{História do Estilo Mamdami}
\pgfdeclareimage[width=2.5cm]{mamdani}{img/mamdani}
\begin{block}{Estilo Mamdani}
	\begin{columns}	
	\begin{column}{7cm}
	O estilo de inferência Mandani foi criado pelo professor Mandani da Universidade de Londres
	em 1975. O seu principal objectivo era desenvolver sistemas Fuzzy, baseando-se em regras de conjuntos Fuzzy 
	com o intuito de representar experiências da vida real.
	\end{column}
	\begin{column}{3cm}
		\begin{figure}
			\centering
			\pgfuseimage{mamdani}
			\caption{Ebrahim Mamdani (1943--2010)}
		\end{figure}		
	\end{column}
	\end{columns}
\end{block}
\end{frame}

\begin{frame}{Estilo Mamdani}
	\begin{definition}{Estilo Mamdani}
		O processo de raciocínio do \alert{Estilo Mamdani} é implementado seguindo as quatro etapas seguintes:
		\begin{enumerate}[<+->]
			\item Fuzzyficação;
			\item Avaliação das Regras Fuzzy;
			\item Agregação das Regras Fuzzy;
			\item Defuzzyficação.
		\end{enumerate}
	\end{definition}
\end{frame}

\begin{frame}{Exercício}
\begin{example}[Análise de Risco]
	Considere a análise de riscos num projecto. 
	Pretende-se estabelecer, sendo conhecidos um valor $x$ de recurso 
	monetário e um número $y$ de funcionários para trabalhar no mesmo,
	qual o risco $z$ nesse projecto.
\end{example}
\end{frame}

\begin{frame}{Variáveis de Entrada}
	\begin{columns}
	\begin{column}{5cm}
	\begin{tabular}{c|c} 
		\hline
		\multicolumn{2}{c}{{\textbf{Fundos do projecto($x$)}}}\\ 
		\hline
		Valor linguístico & Notação \\ \hline
		Inadequado & A1 \\
		Razoável & A2 \\ 
		Adequado & A3 \\ \hline
	\end{tabular}
	\end{column}
	\begin{column}{5cm}
	\begin{tabular}{c|c} 
		\hline
		\multicolumn{2}{c}{{\textbf{Funcionários do Projecto($y$)}}}\\ 
		\hline
		Valor linguístico & Notação \\ \hline
		Pequeno & B1 \\ 
		Grande & B2 \\ \hline
	\end{tabular}
	\end{column}
	\end{columns}
\end{frame}

\begin{frame}{Variáveis de Saída}
\begin{tabular}{c|c} 
		\hline
		\multicolumn{2}{c}{{\textbf{Risco do Projecto($z$)}}}\\ 
		\hline
		Valor linguístico & Notação \\ \hline
		Baixo & C1 \\
		Normal & C2 \\ 
		Alto & C3 \\ \hline
	\end{tabular}
\end{frame}

\begin{frame}{Fuzzyficação}
	\begin{example}[Entradas Crisp]
		Sejam $x$ e $y$ duas entradas crisp representando os conjuntos Fuzzy $X$ (Fundos do Projecto)
		e $Y$ (Funcionários do Projecto respectivamente). Então aplicando as entradas as conjuntos Fuzzy obtemos o valor
		das funções de pertença.
	\end{example}

	
\end{frame}

\begin{frame}{Fuzzyficacao da Variável referente aos Fundos do Projecto}
\pgfdeclareimage[height=4cm]{fuzzyfic1}{img/fuzzyfic1}
	\begin{figure}
			\centering
			\pgfuseimage{fuzzyfic1}
			\caption{Fuzzyficacao da Variável referente aos Fundos do Projecto}
		\end{figure}		
\end{frame}

\begin{frame}{Fuzzyficacao da Variável referente aos Funcionários do Projecto}
\pgfdeclareimage[height=4cm]{fuzzyfic2}{img/fuzzyfic2}
	\begin{figure}
			\centering
			\pgfuseimage{fuzzyfic2}
			\caption{Fuzzyficacao da Variável referente aos Funcionários do Projecto}
		\end{figure}		
\end{frame}

\begin{frame}{Variáveis de Entrada Fuzzificadas}
	\begin{columns}
	\begin{column}{5cm}
	\begin{tabular}{c|c} 
		\hline
		\multicolumn{2}{c}{{\textbf{Fundos do projecto ($x$)}}}\\ 
		\hline
		Etiqueta & Valor \\ \hline
		 A1 & $0.5$ \\
		A2  & $0.2$ \\ 
		A3 & $0$ \\ \hline
	\end{tabular}
	\end{column}
	\begin{column}{5cm}
	\begin{tabular}{c|c} 
		\hline
		\multicolumn{2}{c}{{\textbf{Funcionários do Projecto ($y$)}}}\\ 
		\hline
		Etiqueta & Valor \\ \hline
		 B1 & $0.1$ \\ 
		 B2 & $0.7$ \\ \hline
	\end{tabular}
	\end{column}
	\end{columns}
\end{frame}

\begin{frame}{Avaliação das Regras Fuzzy}
	\begin{example}[Avaliação das Regras Fuzzy]
		Com base nos graus de pertinência e nas correlações entre as variáveis linguísticas, têm-se as regras.
		\begin{enumerate}[<+->]
			\item \alert{IF} (($x$ is $A3 (0)$) \alert{OR} ($y$ is $B1 (0.1)$)) \alert{THEN} ($z$ is $C1 (0.1)$)
			\item IF (($x$ is $A2 (0.2)$) \alert{AND} ($y$ is $B2 (0.7)$)) THEN ($z$ is $C2 (0.2)$)
			\item IF ($x$ is $A1 (0.5)$ THEN ($z$ is $C3 (0.5)$))
		\end{enumerate}
	\end{example}
\end{frame}

\begin{frame}{Agregação das Regras Fuzzy}
	\pgfdeclareimage[height=4cm]{agrega}{img/agrega}
	\begin{figure}
			\centering
			\pgfuseimage{agrega}
			\caption{Conjunto Fuzzy Resultante do Processo de Agregação das Regras Fuzzy}
		\end{figure}		
\end{frame}

\begin{frame}{Defuzzyficação}
	\begin{definition}[Defuzzyficação]
		O método de defuzzyficação mais comum é a técnica do centróide, que 
		obtém o ponto onde uma linha vertical divide ao meio um conjunto agregado.
		A equação que descrever o cálculo da centróide é a seguinte $\mathcal{COG}$:
		$$
			\mathcal{COG} = \frac{\sum_{x=a}^{b}\mu(x) \cdot x}{\sum_{x=a}^{b}\mu(x)}
		$$
	\end{definition}
\end{frame}

\begin{frame}{Defuzzyficação --- Exemplo}
\begin{example}
Considerando o conjunto Fuzzy anterior, o resultado numérico obtido com a aplicação técnica do centróide $\mathcal{COG}$ é dado
por (considerando intervalos percentuais de $10\%$, variando de $0\%$ a $100\%$):
		$$
			\mathcal{COG} = \frac{ \begin{array}{c}
						(0 + 10 + 20) \cdot 0.1 + \\
						(30 + 40 +50) \cdot 0.2 + \\
						(60 + 70 + 80 + 90 + 100)\cdot 0.5 
						\end{array}}
					{\begin{array}{c}
						0.1 + 0.1 + 0.1 + \\
						0.2 + 0.2 + 0.2 + \\
						0.5 + 0.5 + 0.5 + 0.5
					 \end{array}} = 67.4
		$$
Assim tem-se que o risco do projecto em questão é de $67.4\%$.
\end{example}
\end{frame}

\subsection{Bousi-Prolog}

\begin{frame}{Breve História\ldots}
\begin{itemize}[<+->]
	\item[-] Os matemáticos descobriram que apesar da lógica 
	de primeira ordem não ser automaticamente dedutível,
	existem subconjuntos que o são;
	\item[1965] Robinson definiu a dedução automática;
	\item[1969] Green Implementou um sistema de Resolução em Lisp
	\item[1970] Kowalsky começa a usar as Cláusulas de Horn (subconjunto
	da lógica da 1ª ordem) para ``provas automáticas''.
	
\end{itemize}
\end{frame}

\begin{frame}{Breve História\ldots}
	\begin{itemize}[<+->]
		\item[1972] Um grupo de investigadores da Universidade de Marselha desenvolveu 
		um sistema de resolução para as Cláusulas de Horn;
		\item[1980-] O governo Japonês investiu no projecto designado por quinta geração que
		teve como resultado grandes contribuições para a computação lógica;
		\item[2008] Julián-Iranzo, Rubio-Manzano e Gallardo Casero 
	propuseram uma extensão à máquina de inferência Prolog,
	utilizando lógica Fuzzy, para que existissem ``respostas
	mais flexíveis às perguntas''. Para isso foi implementado
	o sistema Bousi~Prolog e continua em desenvolvimento na
	Universidad de Castilla-La Mancha.	
	\end{itemize}
\end{frame}

\begin{frame}{Porquê Prolog?}
\begin{itemize}[<+->]
\item 
	\begin{tabular}{p{3cm}p{5cm}}
{\textbf{Convencional}} & Processamento Numérico \\
{\textbf{Lógica}}  & Processamento Simbólico;
	\end{tabular}
\item 
	\begin{tabular}{p{3cm}p{5cm}}
{\textbf{Convencional}} & Soluções Algorítmicas \\
{\textbf{Lógica}}  & Soluções Heurística;
	\end{tabular}
\item 
	\begin{tabular}{p{3cm}p{5cm}}
{\textbf{Convencional}} & Estruturas de Controle e Conhecimento Integradas  \\
{\textbf{Lógica}}  &  Estruturas de Controle e Conhecimento Separadas.  
	\end{tabular}
\end{itemize}
\end{frame}

\begin{frame}{Porquê Prolog?}
\begin{itemize}[<+->]
\item 
	\begin{tabular}{p{3cm}p{5cm}}
{\textbf{Convencional}} & Difícil Modificação \\
{\textbf{Lógica}}  & Fácil Modificação ;
	\end{tabular}
\item 
	\begin{tabular}{p{3cm}p{5cm}}
{\textbf{Convencional}} & Somente Respostas Totalmente Correctas \\
{\textbf{Lógica}}  & Incluem Respostas Parcialmente Correctas;
	\end{tabular}
\item 
	\begin{tabular}{p{3cm}p{5cm}}
{\textbf{Convencional}} & Somente a Melhor Solução Possível  \\
{\textbf{Lógica}}  &  Incluem Todas as Soluções Possíveis.  
	\end{tabular}
\end{itemize}

\end{frame}

\begin{frame}{Arquitectura Prolog}
\pgfdeclareimage[width=.8\textwidth]{bc}{img/bc}
\begin{figure}
\centering
\pgfuseimage{bc}
\caption{Arquitectura Prolog}
\end{figure}
\end{frame}

\begin{frame}{Nota}
\begin{itemize}
	\item Basicamente um programa Prolog é um conjunto de \alert{Factos} e \alert{Regras};
	\item A interacção é feita através de \alert{Queries}.
\end{itemize}
\end{frame}

\begin{frame}{Exemplo}
	\pgfdeclareimage[height=3.5cm]{familia}{img/familia}
	\begin{example}[Árvore Genealógica]				
		Pretende-se escrever em prolog a árvore genealógica seguinte,
		e representar as relações familiares entre os indivíduos.
		\begin{figure}
			\centering
			\pgfuseimage{familia}
		\end{figure}		
	\end{example}
\end{frame}

\frame[label=showcode]{
\pgfdeclareimage[height=7.5cm]{showcode}{img/showcode}
\begin{figure}
\centering
\pgfuseimage{showcode}
\end{figure}}

\begin{frame}{Programa Heurístico}
	\begin{example}[Alice na Floresta do esquecimento]
		A Alice tinha má memória.
		Um dia entrou na floresta do Esquecimento e esquesceu-se do dia-da-semana.
		Os seus amigos \alert{Coelho} e \alert{Cuco} são visitantes frequentes da floresta. Estes dois
		são criaturas estranhas.\\
		O coelho \alert{mente} às \alert{Segundas}, \alert{Terças} e \alert{Quartas} e diz a \alert{verdade} no \alert{resto da Semana}.
		Por outro lado, o Cuco \alert{mente} às \alert{Quintas}, \alert{Sextas} e \alert{Sábados} \alert{dizendo a verdade no resto dos dias}.
	\end{example}
\end{frame}

\begin{frame}{Programa Heurístico}
	\begin{example}[Alice na Floresta do esquecimento]
		Um certo dia a Alice encontrou estes dois debaixo de uma árvore. Eles fizeram as seguintes declarações:
		\begin{itemize}
			\item Coelho: \alert{Ontem} foi um dos dias que eu \alert{menti};
			\item Cuco: \alert{Ontem} foi um dos dias que eu \alert{menti}.
		\end{itemize}
		
		A Alice foi capaz, usando estas declarações, de deduzir o dia-da-semana em que se encontrava.
	\end{example}
\end{frame}

\begin{frame}{Programa Heurístico}
	\begin{example}[Alice na Floresta do esquecimento]
		Com este exemplo pretende-se: 
		\begin{enumerate}
			\item Escrever uma Base de Conhecimento que descreva esta história;
			\item Escrever um predicado diadehoje/1 que lhe permita saber qual o dia-da-semana.
		\end{enumerate}
	\end{example}
\end{frame}

\againframe{showcode}

\begin{frame}{Extensões Bousi-Prolog}
	\begin{example}[Exemplos Bousi-Prolog]
		Em seguida serão apresentados os seguintes exemplos em Bousi-Prolog
		\begin{itemize}
			\item Programa de cálculo de idades;
			\item Programa de que emula um sistema de ``Information Retrieval'';
			\item Programa de escolha de apartamento inteligente.
		\end{itemize}
	\end{example}
\end{frame}

\againframe{showcode}

\section{E agora?}

\begin{frame}{O que pode ser feito?}
	\begin{itemize}[<+->]
		\item Estudar o sistema que esta a ser desenvolvido na universidade
		de Málaga \alert{FSQL} (Fuzzy SQL) e tentar integrar nos sistemas actuais
		(Neste site propões-se a interacção com SQL Server e Oracle).
		\item Estudar as funções Fuzzy disponibilizadas nos SGBDs actuais 
		(exemplo Soundex e Difference no SQL SERVER).
		\item Estudar os algoritmos de procura de pares em Fuzzy (Busca de informação repetida).
	\end{itemize}
\end{frame}

\begin{frame}{O que há para fazer ainda em Fuzzy?}
	\begin{itemize}[<+->]
		\item Implementar sistemas fuzzy com relações n-árias;
		\item Estudar a possibilidade de desenvolver sistemas com Conjuntos Analógicos;
		\item Modelação de um sistema deductivo de pesquisa de documentação inteligente.
	\end{itemize}
\end{frame}


\section{Sumário}

\begin{frame}{Sumário}
\begin{block}{Recapitulando\ldots}
	\begin{itemize}[<+->]
		\item Vimos motivação do estudo da lógica Fuzzy, bem como
		algumas vantagens da implementação desta;
		\item Estudamos os conceitos Fundamentais da Lógica Fuzzy;
		\item Resolvemos um exercício de Decisão recorrendo 
		à metodologia Mamdani;
		\item Apresentamos o sistema Bousi-Prolog, analizando para
		isso vários exemplos de programas em Prolog e Bousi-Prolog;
		\item Discutimos onde poderia ser usado o conceito na
		nossa corporação e melhorias que poderiam ser feitas a este. 
	\end{itemize}
\end{block}
\end{frame}

\begin{frame}{Dúvidas?}
\pgfdeclareimage[height=4.5cm]{duvidas}{img/duvidas}
		\begin{figure}
			\centering
			\pgfuseimage{duvidas}
		\end{figure}		
\end{frame}

%%%%%%%%%%%%%%%%%%%%%%%%%%%%%%%%%%%%%%%%%%%%%%%%%%%%%%%%%%%%%%%%%%%%%%%%%%%%%

%%%%%%%%%%%%%%%%%%%%%%%%%%%%%%%BIBLIOGRAFIA%%%%%%%%%%%%%%%%%%%%%%%%%%%%%%%%%%%%%%%%%%%%%%
\section*{Referências}

\begin{frame}[allowframebreaks]
  \frametitle<presentation>{Para ler depois}
    
  \begin{thebibliography}{10}
    
  \beamertemplatebookbibitems
  \bibitem{Tanaka1996}
    Kazuo Tanaka.
    \newblock {\em An Introduction to Fuzzy Logic for Practical Applications}.
    \newblock Springer, 1996.

  \bibitem{Shapiro1986}
    Shapiro.
    \newblock {\em The Art of Prolog}.
    \newblock MIT Press, 1986.
 
  \beamertemplatearticlebibitems
  % Followed by interesting articles. Keep the list short. 

  \bibitem{Zadeh1965}
    Zadeh.
    \newblock Fuzzy Sets.
    \newblock {\em Information and Control}, 8(3):338--353,
    1965.
  \end{thebibliography}
\end{frame}

\appendix
\section{\appendixname}
\frame{\tableofcontents}
\subsection{Material Adicional}

\frame[label=eqpertencadetail]{
  \frametitle{Definições Fuzzy --- Etiquetas e Conjuntos Fuzzy Detalhe}

$$
		\mu_X(x_0) = \left \{ \begin{array}{ll}
				$1$ & \textrm{sse $x \in \mathcal{C}$} \\
				$0$ & \textrm{sse $x \notin \mathcal{C}$} \\
				$0$ \le \mu_X(x_0) \le 1 & \textrm{sse $x \sim \in \mathcal{C}$} \\
			\end{array} \right . 
		$$	

    \hfill\hyperlink{eqpertenca}{\beamerreturnbutton{Voltar}}
}

\end{document}



\end{verbatim}

\item
  Create an extra file named, say, |main.article.tex| with the following content:

\begin{verbatim}
\documentclass{article}
\usepackage{beamerarticle}
\setjobnamebeamerversion{main.beamer}
% $Header: /home/vedranm/bitbucket/beamer/solutions/conference-talks/conference-ornate-20min.en.tex,v 90e850259b8b 2007/01/28 20:48:30 tantau $

\documentclass[portuges]{beamer}

\mode<presentation>
{
  \usetheme{Warsaw}
  \setbeamercovered{transparent}
}

\usepackage{babel}

\usepackage[utf8x]{inputenc}

\usepackage{times}
\usepackage[T1]{fontenc}
\usepackage{amsmath}
\usepackage{latexsym}
%\usepackage{amstex}

\title[Fuzzy \& SIAD] %
{Lógica Fuzzy aplicada a \\Sistemas de Informação de Apoio à Decisão}


\author[António Sérgio Matos da Silva] %
{António Sérgio Matos da Silva \\ \texttt{an.silva@logica.com}}

\institute[Logica] % 
{
  Telecommunication Business\\
  Logica
}

\date[Reunião Mensal Zon] % 
{\today\\Reunião Mensal Zon}

\subject{Theoretical Computer Science}


% Delete this, if you do not want the table of contents to pop up at
% the beginning of each subsection:
\AtBeginSubsection[]
{
  \begin{frame}<beamer>{Programa}
    \tableofcontents[currentsection,currentsubsection]
  \end{frame}
}

\pgfdeclareimage[height=0.5cm]{logo}{img/logo}
\logo{\pgfuseimage{logo}}

%\beamerdefaultoverlayspecification{<+->}

%%%%%%%%%%%%%%%%%%%%%%%%%%%%%%%%%%%%MY SPECS
\newtheorem{convention}[theorem]{Convenção}
\newtheorem{proposition}[theorem]{Proposição}

\begin{document}

\begin{frame}
  \titlepage
\end{frame}

\begin{frame}
\begin{quote}
``À medida que a complexidade aumenta, 
as declarações precisas perdem relevância e as declarações relevantes perdem precisão.'' 
\end{quote}
\begin{quote}
Lofti Zadeh
\end{quote}
\end{frame}

\begin{frame}{Programa}
  \tableofcontents[pausesections]
\end{frame}
%%%%%%%%%%%%%%%%%%%%%%%%%%%%%%APRESENTACAO%%%%%%%%%%%%%%%%%%%%%%%%%%%%%%%%

\section{Breve contextualização Teórica}
\subsection{Motivação}

\begin{frame}{Como ``Raciocina'' um Computador Tradicional?}
\pgfdeclareimage[width=.6\textwidth]{pcthink}{img/pcthink}
\begin{figure}
\centering
\pgfuseimage{pcthink}
\caption{Funcionamento de um PC}
\end{figure}
\end{frame}


\begin{frame}{Pensamento Lógico Humano}
\pgfdeclareimage[width=.6\textwidth]{larrythink}{img/larrythink}
\begin{figure}
\centering
\pgfuseimage{larrythink}
\caption{Funcionamento do cérebro de Larry Wall}
\end{figure}
\end{frame}

\begin{frame}{Motivação}
\begin{itemize}[<+->]
	\item Tradicionalmente, o comportamento dos sistemas é realizado
		pela representação da lógica de Boole, aceitando apenas dois resultados
		\alert{zero ou um}, \alert{verdadeiro ou falso}, tudo ou nada.
	\item No entanto, em muitas situações não relevamos esta dicotomia comportamental.
		\begin{itemize}
			\item Apesar de à partida, uma pessoa com $1.75\ m$ ser alta, esta não é assim
			tão alta;
			\item Quando alguém nos diz que nos ama, \alert{não sabemos o quanto} esta nos ama;
			\item Nem sempre precisamos de obter um resultado baseado numa \alert{certeza},
			ficamos satisfeitos apenas com um certo grau de \alert{confiança}.
		\end{itemize}		 
\end{itemize}

\end{frame}

\begin{frame}{Analogia}
	\pgfdeclareimage[height=2.5cm]{rightcurve}{img/rightcurve}
	\begin{example}
			\begin{figure}
			\centering
			\pgfuseimage{rightcurve}
			\end{figure}

			Pretende-se comparar os dois mecanismos
			para efectuar a curva à direita.		 	
	\end{example}
\end{frame}
\begin{frame}{Analogia}
\begin{columns}[T]
  		\begin{column}{5cm}
		    \begin{block}{Booleana}<1->  				
				\begin{enumerate}[<+->]
					\item Pressionar o travão com uma força de $20\ Newtons$.
					\item Inclinar o volante $15\ graus$ para a Direita
					\item Colocar o volante na posição inicial ($0\ graus$).
				\end{enumerate}		
			\end{block}
		  \end{column}
		  \begin{column}{5cm}
		    \begin{block}{Humana}<4->
  			\begin{enumerate}[<+->]
					\item Reduza a velocidade.

					\item Vire \alert{um pouco} para a direita
					\item Vire \alert{mais um pouco} para a direita
					\item Siga em frente.
				\end{enumerate}		
		    \end{block}

		  \end{column}
		\end{columns}

\end{frame}


\subsection{História e Uso da Lógica Fuzzy}

\begin{frame}{Resumo Histórico\ldots}

	\pgfdeclareimage[height=3cm]{lukasiewicz}{img/Lukasiewicz}
   \begin{columns}
	\begin{column}{7cm}
	\begin{itemize}[<+->]
		\item[1930] Jan Lukasiewz propôs o estudo de termos qualitativos do tipo alto, velho e quente 
		e propôs a idéia da utilização de um intervalo de valores entre $0$ e $1$ para descrever a veracidade
		de uma dada afirmação;
		\item[1937] Max Black definiu o primeiro conjunto fuzzy e descreveu algumas idéias básicas de operações com estes.
	\end{itemize}
	\end{column}
	\begin{column}{4cm}
		\begin{figure}
			\centering
			\pgfuseimage{lukasiewicz}
			\caption{Jan Lukasiewicz (1878--1956)}
		\end{figure}
	\end{column}	
   \end{columns}
\end{frame}

\begin{frame}{Resumo Histórico\ldots}
\pgfdeclareimage[height=3cm]{zadeh}{img/zadeh}
   \begin{columns}
	\begin{column}{7cm}
	\begin{itemize}[<+->]
		\item[1965] Lofti Zadeh publicou o artigo Fuzzy Sets que ficou
		conhecido como a origem da Lógica Fuzzy. Zadeh é conhecido como
		o ``mestre'' da Lógica Fuzzy.
	\end{itemize}
	\end{column}
	\begin{column}{4cm}
		\begin{figure}
			\centering
			\pgfuseimage{zadeh}
			\caption{Lofti Zadeh (1921--Actualidade)}
		\end{figure}
	\end{column}	
   \end{columns}
\end{frame}

\begin{frame}{Uso da Lógica Fuzzy}
\begin{itemize}
	\item[1970] Primeira aplicação da Lógica Fuzzy na engenharia de controlo;
	\item[1975] Introdução da Lógica Fuzzy no Japão;
        \item[1985] Ampla utilização no Japão;
        \item[1990] Ampla utilização na Europa;
        \item[1995] Ampla utilização nos EUA;
        \item[1996] 1100 aplicações com Lógica Fuzzy publicadas; 
	\item[2000] Aplicada a finanças e controle multi-variável.
\end{itemize}
\end{frame}

\begin{frame}{Porquê usar Fuzzy?}
\begin{block}{Vantagens da Lógica Fuzzy}
\begin{itemize}
	\item<1-> Robusta porque não requer entradas precisas;
	\item<2-> Facilmente modificável pois é baseada em regras;
	\item<3-> Evita o formalismo matemático para sistemas não lineares;
	\item<3-> Solução rápida e barata para sistemas complexos não lineares;
	\item<4-> Implementável em microprocessadores.
\end{itemize}
\end{block}
\end{frame}


\subsection{Fundamentos da Lógica Fuzzy}

\begin{frame}{Conceito de Difusidade}
	\pgfdeclareimage[height=3cm]{arcond}{img/arcond}  	
	\begin{figure}
			\centering
			\pgfuseimage{arcond}
	\end{figure}
		
	\begin{example}[O caso do ar condicionado]
  		No edifício Pinta o ar condicionado encontra-se constantemente
		avariado. Pretende-se desenvolver uma lógica que pretende 
		avaliar se uma dada temperatura é \alert{confortável} ou não.
	\end{example}
\end{frame}

\begin{frame}{Conceito de Difusidade}
	\pgfdeclareimage[width=.6\textwidth]{fuzzy1}{img/fuzzy1}	
	\begin{quote}
		Numa primeira análise vem o José e diz que a temperatura ideal
		para ele é de exactamente $20º\ C$.
	\end{quote}
	\begin{figure}
			\centering
			\pgfuseimage{fuzzy1}
			\caption{Função de Verdade de uma Lógica tipicamente Booleana.}
	\end{figure}
\end{frame}

\begin{frame}{Conceito de Difusidade}
	\pgfdeclareimage[width=.5\textwidth]{fuzzy2}{img/fuzzy2}	
	\begin{quote}
		No entanto o Miguel discorda com valor estipulado anteriormente. ``Vamos
		lá ver! Se a temperatura for 18 ou 22 não deixa de ser bom também!'' 
	\end{quote}
	\begin{figure}
			\centering
			\pgfuseimage{fuzzy2}
			\caption{Função de Verdade de uma Lógica na Cabeça do ``Miguel''.}
	\end{figure}
\end{frame}

\begin{frame}{Conceito de Difusidade}
	\pgfdeclareimage[width=.4\textwidth]{fuzzy3}{img/fuzzy3}	
	\begin{quote}
		No dia seguinte já com o ar condicionado a funcionar e uma
		temperatura a funcionar o Miguel pergunta ao Pedro.\\
		``Bom dia! Está bom tempo hoje!''\\
		Ao que este responde:\\
		``Sim está. \alert{Nem muito quente nem muito frio}\ldots''  
	\end{quote}
	\begin{figure}
			\centering
			\pgfuseimage{fuzzy3}
			\caption{Função de Verdade de uma Lógica Fuzzy.}
	\end{figure}
\end{frame}

\begin{frame}{Conceito de Difusidade}
	\pgfdeclareimage[width=.6\textwidth]{fuzzy3_1}{img/fuzzy3_1}	
	\begin{quote}
		Na verdade o que o Pedro pensou\ldots
	\end{quote}
	\begin{figure}
			\centering
			\pgfuseimage{fuzzy3_1}
			\caption{Função de Verdade de uma Lógica Fuzzy representando mais do que uma valoração ao mesmo tempo.}
	\end{figure}
\end{frame}

\begin{frame}{Definições Fuzzy --- Etiqueta linguística}
	\begin{definition}
		A descrição de inúmeras situações concretas referentes a uma dada variável faz-se por intermédio de \alert{etiquetas
		linguísticas}. Estas representam o carácter \alert{qualitativo} de todas as possibilidades de uma dada variável.
	\end{definition}
	
	\begin{exampleblock}{Etiquetas Bivalentes}<2->
		\begin{itemize}
			\item Lento;
			\item Rápido.
		\end{itemize}	
	\end{exampleblock}
\end{frame}

\begin{frame}{Definições Fuzzy --- Exemplo de Etiquetas Linguísticas}
	\begin{exampleblock}{Etiquetas Trivalentes}
		\begin{itemize}
			\item Baixo;
			\item Médio;
			\item Alto.
		\end{itemize}	
	\end{exampleblock}
	
	\begin{exampleblock}{Etiquetas Multivalentes}
		\begin{itemize}
			\item Muito Frio
			\item Frio;
			\item Moderado;
			\item Quente;
			\item Muito Quente.
		\end{itemize}	
	\end{exampleblock}
\end{frame}

\begin{frame}{Definições Fuzzy --- Variável Linguística}
		\pgfdeclareimage[width=.6\textwidth]{continuousfuzzy}{img/fuzzy3}	
	\begin{definition}<1->
		Uma variável linguística pode ser contínua ou discreta.
		\end{definition}
	\begin{example}<2->[Temperatura]	
	\begin{figure}
			\centering
			\pgfuseimage{continuousfuzzy}
			\caption{Exemplo de Variável Linguística contínua.}
	\end{figure}
	\end{example}
\end{frame}

\begin{frame}
	\pgfdeclareimage[width=.6\textwidth]{discretefuzzy}{img/fuzzydiscrete}
	\begin{example}[Rodas de um camião]
	\begin{figure}
			\centering
			\pgfuseimage{discretefuzzy}
			\caption{Exemplo de Variável Linguística Discreta.}
	\end{figure}	
	\end{example}
\end{frame}

\begin{frame}{Definições Fuzzy --- Conjuntos Fuzzy no Mundo Real}
	\pgfdeclareimage[width=.4\textwidth]{realworldfuzzy}{img/realworldfuzzy}
	\begin{convention}
		Geralmente os Sistemas de Informação baseados num sistema de decisão Fuzzy
		usam a seguinte simplificação das etiquetas. A partir de agora usaremos também
		esta notação.
	\end{convention}

	\begin{example}[Conjunto Fuzzy no Mundo Real]
	\begin{figure}
			\centering
			\pgfuseimage{realworldfuzzy}
			\caption{Exemplo de Variável Linguística Contínua do Mundo Real.}
	\end{figure}	
	\end{example}
\end{frame}

\frame[label=eqpertenca]
{
  \frametitle{Definições Fuzzy --- Etiquetas e Conjuntos Fuzzy}
	\pgfdeclareimage[width=.4\textwidth]{pertenca}{img/pertenca}  	
	\begin{definition}
		Seja $X$ uma Etiqueta Linguística. Esta é representada por um Conjunto Fuzzy $\mathcal{C}$ descrito pela
		função de pertença $\mu_X(x_0)$.
	\end{definition}
		
	\begin{figure}
			\centering
			\pgfuseimage{pertenca}
			\caption{Etiqueta Linguística e respectiva Função Pertença.}
	\end{figure}		

    \hyperlink{eqpertencadetail}{\beamergotobutton{Equação Detalhe}}
}

\begin{frame}{Definições Fuzzy --- Suporte}
\pgfdeclareimage[width=.4\textwidth]{support}{img/support}  	
\begin{definition}
		Seja $X$ uma Etiqueta Linguística. Designa-se por Suporte de $X$ ($\mathcal{S}_X$)
		a zona em que a sua função pertença não é nula. Ou seja: $\{\mu_X(x_0) \ \vert \ \mu_X(x_0) \neq 0 \land x_0 \in [0,1] \}$
	\end{definition}

	\begin{figure}
			\centering
			\pgfuseimage{support}
			\caption{Suporte de uma Etiqueta Linguística}
	\end{figure}
\end{frame}

\begin{frame}{Definições Fuzzy --- Núcleo}
\pgfdeclareimage[width=.4\textwidth]{nucleo}{img/nucleo}  	
\begin{definition}
		Seja $X$ uma Etiqueta Linguística. Designa-se por Núcleo de $X$ ($\mathcal{N}_X$)
		a zona em que a sua função pertença é máxima. Ou seja: $\{\mu_X(x_0) \ \vert \ \mu_X(x_0) = 1 \land x_0 \in [0,1] \}$
	\end{definition}

	\begin{figure}
			\centering
			\pgfuseimage{nucleo}
			\caption{Núcleo de uma Etiqueta Linguística}
	\end{figure}
\end{frame}

\begin{frame}{Definições Fuzzy --- Variável Linguística Normada}  	
\pgfdeclareimage[width=.5\textwidth]{notnormada}{img/notnormada}  	
\begin{definition}
		Uma Variável Línguística diz-se Normada se esta gera um discurso limitado pelo 
		conjunto $[-1,1]$
	\end{definition}
		\begin{figure}
			\centering
			\pgfuseimage{notnormada}
			\caption{Variável não normada.}
	\end{figure}
\end{frame}

\begin{frame}{Definições Fuzzy --- Variável Linguística Normada Exemplo}  	
\pgfdeclareimage[width=.6\textwidth]{normada}{img/normada}  	
	\begin{example}[Variável Linguística Normada]		
	\begin{figure}
			\centering
			\pgfuseimage{normada}
			\caption{Variável normada à escala 1/120.}
	\end{figure}
	\end{example}
\end{frame}

\begin{frame}{Operações Com Conjuntos Fuzzy}
	\begin{proposition}
		Seja $\mho$ o Universo de uma Variável Linguística
		e sejam $\mathcal{A}$ e $\mathcal{B}$ dois Conjuntos Fuzzy 
		definidos por:
		$$
			\begin{array}{ccc}
				\mathcal{A} & = & \{ (x,\mu_\mathcal{A}(x))\ \vert\ x \in \mho \land \mu_\mathcal{A}(x) \in [0,1] \} \\
				\mathcal{B} & = & \{ (x,\mu_\mathcal{B}(x))\ \vert\ x \in \mho \land \mu_\mathcal{B}(x) \in [0,1] \}
			\end{array}
		$$   
	\end{proposition}
\end{frame}

\begin{frame}{Operações Com Conjuntos Fuzzy --- União}
	\pgfdeclareimage[width=.4\textwidth]{orvenn}{img/orvenn}  	

	\begin{definition}[União]
	A união entre $\mathcal{A}$ e $\mathcal{B}$ é definida por:
	$$	
	\mathcal{A} \cup \mathcal{B} = \{ ( x , max(\mu_\mathcal{A}(x),\mu_\mathcal{B}(x)))\ \vert\ x \in \mho \} 
	$$
	\end{definition}	
\begin{figure}
			\centering
			\pgfuseimage{orvenn}
			\caption{Diagrama de Venn da União de dois Conjuntos}
	\end{figure}	
\end{frame}

\begin{frame}{Operações Com Conjuntos Fuzzy --- Intersecção}
	\pgfdeclareimage[width=.4\textwidth]{andvenn}{img/andvenn}  	

	\begin{definition}[Intersecção]
	A Intersecção entre $\mathcal{A}$ e $\mathcal{B}$ é definida por:
	$$	
	\mathcal{A} \cap \mathcal{B} = \{ ( x , min(\mu_\mathcal{A}(x),\mu_\mathcal{B}(x)))\ \vert\ x \in \mho \} 
	$$
	\end{definition}	
\begin{figure}
			\centering
			\pgfuseimage{andvenn}
			\caption{Diagrama de Venn da Intersecção de dois Conjuntos}
	\end{figure}	
\end{frame}

\begin{frame}{Operações Com Conjuntos Fuzzy --- Complemento}
	\pgfdeclareimage[width=.25\textwidth]{notvenn}{img/notvenn}  	

	\begin{definition}[Complemento]
	O Complemento de $\mathcal{A}$ é definido por:
	$$	
	\lnot\mathcal{A} = \{ ( x , \mu_{\lnot\mathcal{A}}(x) )\ \vert\ x \in \mho \land \mu_{\lnot\mathcal{A}}(x) = 1 - \mu_{\mathcal{A}}(x) \} 
	$$
	\end{definition}	
\begin{figure}
			\centering
			\pgfuseimage{notvenn}
			\caption{Diagrama de Venn do Complemento de um Conjunto}
	\end{figure}	
\end{frame}

\begin{frame}{Modificadores Linguísticos}
	\begin{definition}
		Seja $\mathcal{A}$ um conjunto Fuzzy intervalar com a função de pertinência $\mu_\mathcal{A}{x}$.
		Então o \emph{Modificador Línguístico} de $\mathcal{A}$ é uma função intervalar $\mathcal{M}$ definida por:
		$$\mathcal{M}:\mathcal{I}[0,1] \to \mathcal{I}[0,1] $$
		que age na função pertinência $\mu_{\mathcal{I}\mathcal{A}}{x}$ transformando-a em $\mu_{m\mathcal{I}\mathcal{A}}{x}$ onde:
		$$\mu_{m\mathcal{I}\mathcal{A}}(x) = \mathcal{M}(\mu_{\mathcal{I}\mathcal{A}}(x))$$
	\end{definition}
\end{frame}

\begin{frame}{Modificadores Linguísticos --- Very}
	\pgfdeclareimage[height=3cm]{muito}{img/muito}  		
	\begin{definition}[Very]
		O modificador \alert{Very} (muito) define-se por:
		$$\mu_{v\mathcal{A}}{x} = \mu_{v\mathcal{A}}^2{x}$$ 
	\end{definition}
	\begin{figure}
			\centering
			\pgfuseimage{muito}
			\caption{Representação Gráfica do Modificador Very}
	\end{figure}	
\end{frame}

\begin{frame}{Modificadores Linguísticos --- Somewhat}
	\pgfdeclareimage[height=3cm]{pouco}{img/pouco}  		
	\begin{definition}[Somewhat]
		O modificador \alert{Somewhat} (pouco) define-se por:
		$$\mu_{v\mathcal{A}}{x} = \sqrt[2]{\mu_{v\mathcal{A}}{x}}$$ 
	\end{definition}
	\begin{figure}
			\centering
			\pgfuseimage{pouco}
			\caption{Representação Gráfica do Modificador Somewhat}
	\end{figure}	
\end{frame}

\begin{frame}{Modificadores Linguísticos --- Above}
	\pgfdeclareimage[height=3cm]{acima}{img/acima}  		
	\begin{definition}[Above]
		O modificador \alert{Above} (acima) define-se por:
		$$\mu_{v\mathcal{A}}{x} =\mu_{v\mathcal{A}}{x} - \delta$$ 
	\end{definition}
	\begin{figure}
			\centering
			\pgfuseimage{acima}
			\caption{Representação Gráfica do Modificador Above}
	\end{figure}	
\end{frame}

\begin{frame}{Modificadores Linguísticos --- Below}
	\pgfdeclareimage[height=3cm]{abaixo}{img/abaixo}  		
	\begin{definition}[Below]
		O modificador \alert{Below} (abaixo) define-se por:
		$$\mu_{v\mathcal{A}}{x} = \mu_{v\mathcal{A}}{x} + \delta$$ 
	\end{definition}
	\begin{figure}
			\centering
			\pgfuseimage{abaixo}
			\caption{Representação Gráfica do Modificador Below}
	\end{figure}	
\end{frame}

\begin{frame}{Modificadores Linguísticos --- Not}
	\pgfdeclareimage[height=3cm]{nao}{img/nao}  		
	\begin{definition}[Not]
		O modificador \alert{Not} (não) define-se por:
		$$\mu_{v\mathcal{A}}{x} = 1 - \mu_{v\mathcal{A}}{x}$$ 
	\end{definition}
	\begin{figure}
			\centering
			\pgfuseimage{nao}
			\caption{Representação Gráfica do Modificador Not}
	\end{figure}	
\end{frame}

\begin{frame}{Modificadores Linguísticos --- Not Very}
	\pgfdeclareimage[height=3cm]{naomuito}{img/naomuito}  		
	\begin{definition}[Not Very]
		O modificador \alert{Not Very} (não muito) define-se por:
		$$\mu_{v\mathcal{A}}{x} = 1 - \mu_{v\mathcal{A}}^2{x}$$ 
	\end{definition}
	\begin{figure}
			\centering
			\pgfuseimage{naomuito}
			\caption{Representação Gráfica do Modificador Not Very}
	\end{figure}	
\end{frame}

\begin{frame}{Inferência}
\begin{block}{Interferência}
\begin{itemize}[<+->]
\item
Para fazer deduções com conjuntos difusos utilizam-se \alert{regras de inferência},
		formatando afirmações condicionais como implicações do tipo ``if-then'';
\item O antecedente
		diz respeito às ``condições lógicas'' impostas sobre essa variável linguística;
\item O consequente
		diz respeito às ``acções'' decorrentes dessas condições na variável de saída;
\item No controlo difuso costumam haver múltiplas regras de inferência, de acordo com a natureza 
dos estados medidos no processo.
\end{itemize}
\end{block}
\end{frame}

\begin{frame}{Inferência}
\begin{example}[Base de Regras]
$$
	\begin{array}{ccccc}
	if & (antecedente_1) & then & (consequente_1) & or \\
	if & (antecedente_2) & then & (consequente_2) & or \\
	 &  & ... &  & \\
	if & (antecedente_n) & then & (consequente_n) &
	\end{array}
$$
\end{example}
\end{frame}

\begin{frame}{Pausa\ldots}
\pgfdeclareimage[width=0.9\textwidth]{kitkat}{img/kit_kat}  	
\begin{figure}
			\centering
			\pgfuseimage{kitkat}
	\end{figure}		
\end{frame}

\section{Estado da Arte}
\subsection{Estilo Mamdami}

\begin{frame}{História do Estilo Mamdami}
\pgfdeclareimage[width=2.5cm]{mamdani}{img/mamdani}
\begin{block}{Estilo Mamdani}
	\begin{columns}	
	\begin{column}{7cm}
	O estilo de inferência Mandani foi criado pelo professor Mandani da Universidade de Londres
	em 1975. O seu principal objectivo era desenvolver sistemas Fuzzy, baseando-se em regras de conjuntos Fuzzy 
	com o intuito de representar experiências da vida real.
	\end{column}
	\begin{column}{3cm}
		\begin{figure}
			\centering
			\pgfuseimage{mamdani}
			\caption{Ebrahim Mamdani (1943--2010)}
		\end{figure}		
	\end{column}
	\end{columns}
\end{block}
\end{frame}

\begin{frame}{Estilo Mamdani}
	\begin{definition}{Estilo Mamdani}
		O processo de raciocínio do \alert{Estilo Mamdani} é implementado seguindo as quatro etapas seguintes:
		\begin{enumerate}[<+->]
			\item Fuzzyficação;
			\item Avaliação das Regras Fuzzy;
			\item Agregação das Regras Fuzzy;
			\item Defuzzyficação.
		\end{enumerate}
	\end{definition}
\end{frame}

\begin{frame}{Exercício}
\begin{example}[Análise de Risco]
	Considere a análise de riscos num projecto. 
	Pretende-se estabelecer, sendo conhecidos um valor $x$ de recurso 
	monetário e um número $y$ de funcionários para trabalhar no mesmo,
	qual o risco $z$ nesse projecto.
\end{example}
\end{frame}

\begin{frame}{Variáveis de Entrada}
	\begin{columns}
	\begin{column}{5cm}
	\begin{tabular}{c|c} 
		\hline
		\multicolumn{2}{c}{{\textbf{Fundos do projecto($x$)}}}\\ 
		\hline
		Valor linguístico & Notação \\ \hline
		Inadequado & A1 \\
		Razoável & A2 \\ 
		Adequado & A3 \\ \hline
	\end{tabular}
	\end{column}
	\begin{column}{5cm}
	\begin{tabular}{c|c} 
		\hline
		\multicolumn{2}{c}{{\textbf{Funcionários do Projecto($y$)}}}\\ 
		\hline
		Valor linguístico & Notação \\ \hline
		Pequeno & B1 \\ 
		Grande & B2 \\ \hline
	\end{tabular}
	\end{column}
	\end{columns}
\end{frame}

\begin{frame}{Variáveis de Saída}
\begin{tabular}{c|c} 
		\hline
		\multicolumn{2}{c}{{\textbf{Risco do Projecto($z$)}}}\\ 
		\hline
		Valor linguístico & Notação \\ \hline
		Baixo & C1 \\
		Normal & C2 \\ 
		Alto & C3 \\ \hline
	\end{tabular}
\end{frame}

\begin{frame}{Fuzzyficação}
	\begin{example}[Entradas Crisp]
		Sejam $x$ e $y$ duas entradas crisp representando os conjuntos Fuzzy $X$ (Fundos do Projecto)
		e $Y$ (Funcionários do Projecto respectivamente). Então aplicando as entradas as conjuntos Fuzzy obtemos o valor
		das funções de pertença.
	\end{example}

	
\end{frame}

\begin{frame}{Fuzzyficacao da Variável referente aos Fundos do Projecto}
\pgfdeclareimage[height=4cm]{fuzzyfic1}{img/fuzzyfic1}
	\begin{figure}
			\centering
			\pgfuseimage{fuzzyfic1}
			\caption{Fuzzyficacao da Variável referente aos Fundos do Projecto}
		\end{figure}		
\end{frame}

\begin{frame}{Fuzzyficacao da Variável referente aos Funcionários do Projecto}
\pgfdeclareimage[height=4cm]{fuzzyfic2}{img/fuzzyfic2}
	\begin{figure}
			\centering
			\pgfuseimage{fuzzyfic2}
			\caption{Fuzzyficacao da Variável referente aos Funcionários do Projecto}
		\end{figure}		
\end{frame}

\begin{frame}{Variáveis de Entrada Fuzzificadas}
	\begin{columns}
	\begin{column}{5cm}
	\begin{tabular}{c|c} 
		\hline
		\multicolumn{2}{c}{{\textbf{Fundos do projecto ($x$)}}}\\ 
		\hline
		Etiqueta & Valor \\ \hline
		 A1 & $0.5$ \\
		A2  & $0.2$ \\ 
		A3 & $0$ \\ \hline
	\end{tabular}
	\end{column}
	\begin{column}{5cm}
	\begin{tabular}{c|c} 
		\hline
		\multicolumn{2}{c}{{\textbf{Funcionários do Projecto ($y$)}}}\\ 
		\hline
		Etiqueta & Valor \\ \hline
		 B1 & $0.1$ \\ 
		 B2 & $0.7$ \\ \hline
	\end{tabular}
	\end{column}
	\end{columns}
\end{frame}

\begin{frame}{Avaliação das Regras Fuzzy}
	\begin{example}[Avaliação das Regras Fuzzy]
		Com base nos graus de pertinência e nas correlações entre as variáveis linguísticas, têm-se as regras.
		\begin{enumerate}[<+->]
			\item \alert{IF} (($x$ is $A3 (0)$) \alert{OR} ($y$ is $B1 (0.1)$)) \alert{THEN} ($z$ is $C1 (0.1)$)
			\item IF (($x$ is $A2 (0.2)$) \alert{AND} ($y$ is $B2 (0.7)$)) THEN ($z$ is $C2 (0.2)$)
			\item IF ($x$ is $A1 (0.5)$ THEN ($z$ is $C3 (0.5)$))
		\end{enumerate}
	\end{example}
\end{frame}

\begin{frame}{Agregação das Regras Fuzzy}
	\pgfdeclareimage[height=4cm]{agrega}{img/agrega}
	\begin{figure}
			\centering
			\pgfuseimage{agrega}
			\caption{Conjunto Fuzzy Resultante do Processo de Agregação das Regras Fuzzy}
		\end{figure}		
\end{frame}

\begin{frame}{Defuzzyficação}
	\begin{definition}[Defuzzyficação]
		O método de defuzzyficação mais comum é a técnica do centróide, que 
		obtém o ponto onde uma linha vertical divide ao meio um conjunto agregado.
		A equação que descrever o cálculo da centróide é a seguinte $\mathcal{COG}$:
		$$
			\mathcal{COG} = \frac{\sum_{x=a}^{b}\mu(x) \cdot x}{\sum_{x=a}^{b}\mu(x)}
		$$
	\end{definition}
\end{frame}

\begin{frame}{Defuzzyficação --- Exemplo}
\begin{example}
Considerando o conjunto Fuzzy anterior, o resultado numérico obtido com a aplicação técnica do centróide $\mathcal{COG}$ é dado
por (considerando intervalos percentuais de $10\%$, variando de $0\%$ a $100\%$):
		$$
			\mathcal{COG} = \frac{ \begin{array}{c}
						(0 + 10 + 20) \cdot 0.1 + \\
						(30 + 40 +50) \cdot 0.2 + \\
						(60 + 70 + 80 + 90 + 100)\cdot 0.5 
						\end{array}}
					{\begin{array}{c}
						0.1 + 0.1 + 0.1 + \\
						0.2 + 0.2 + 0.2 + \\
						0.5 + 0.5 + 0.5 + 0.5
					 \end{array}} = 67.4
		$$
Assim tem-se que o risco do projecto em questão é de $67.4\%$.
\end{example}
\end{frame}

\subsection{Bousi-Prolog}

\begin{frame}{Breve História\ldots}
\begin{itemize}[<+->]
	\item[-] Os matemáticos descobriram que apesar da lógica 
	de primeira ordem não ser automaticamente dedutível,
	existem subconjuntos que o são;
	\item[1965] Robinson definiu a dedução automática;
	\item[1969] Green Implementou um sistema de Resolução em Lisp
	\item[1970] Kowalsky começa a usar as Cláusulas de Horn (subconjunto
	da lógica da 1ª ordem) para ``provas automáticas''.
	
\end{itemize}
\end{frame}

\begin{frame}{Breve História\ldots}
	\begin{itemize}[<+->]
		\item[1972] Um grupo de investigadores da Universidade de Marselha desenvolveu 
		um sistema de resolução para as Cláusulas de Horn;
		\item[1980-] O governo Japonês investiu no projecto designado por quinta geração que
		teve como resultado grandes contribuições para a computação lógica;
		\item[2008] Julián-Iranzo, Rubio-Manzano e Gallardo Casero 
	propuseram uma extensão à máquina de inferência Prolog,
	utilizando lógica Fuzzy, para que existissem ``respostas
	mais flexíveis às perguntas''. Para isso foi implementado
	o sistema Bousi~Prolog e continua em desenvolvimento na
	Universidad de Castilla-La Mancha.	
	\end{itemize}
\end{frame}

\begin{frame}{Porquê Prolog?}
\begin{itemize}[<+->]
\item 
	\begin{tabular}{p{3cm}p{5cm}}
{\textbf{Convencional}} & Processamento Numérico \\
{\textbf{Lógica}}  & Processamento Simbólico;
	\end{tabular}
\item 
	\begin{tabular}{p{3cm}p{5cm}}
{\textbf{Convencional}} & Soluções Algorítmicas \\
{\textbf{Lógica}}  & Soluções Heurística;
	\end{tabular}
\item 
	\begin{tabular}{p{3cm}p{5cm}}
{\textbf{Convencional}} & Estruturas de Controle e Conhecimento Integradas  \\
{\textbf{Lógica}}  &  Estruturas de Controle e Conhecimento Separadas.  
	\end{tabular}
\end{itemize}
\end{frame}

\begin{frame}{Porquê Prolog?}
\begin{itemize}[<+->]
\item 
	\begin{tabular}{p{3cm}p{5cm}}
{\textbf{Convencional}} & Difícil Modificação \\
{\textbf{Lógica}}  & Fácil Modificação ;
	\end{tabular}
\item 
	\begin{tabular}{p{3cm}p{5cm}}
{\textbf{Convencional}} & Somente Respostas Totalmente Correctas \\
{\textbf{Lógica}}  & Incluem Respostas Parcialmente Correctas;
	\end{tabular}
\item 
	\begin{tabular}{p{3cm}p{5cm}}
{\textbf{Convencional}} & Somente a Melhor Solução Possível  \\
{\textbf{Lógica}}  &  Incluem Todas as Soluções Possíveis.  
	\end{tabular}
\end{itemize}

\end{frame}

\begin{frame}{Arquitectura Prolog}
\pgfdeclareimage[width=.8\textwidth]{bc}{img/bc}
\begin{figure}
\centering
\pgfuseimage{bc}
\caption{Arquitectura Prolog}
\end{figure}
\end{frame}

\begin{frame}{Nota}
\begin{itemize}
	\item Basicamente um programa Prolog é um conjunto de \alert{Factos} e \alert{Regras};
	\item A interacção é feita através de \alert{Queries}.
\end{itemize}
\end{frame}

\begin{frame}{Exemplo}
	\pgfdeclareimage[height=3.5cm]{familia}{img/familia}
	\begin{example}[Árvore Genealógica]				
		Pretende-se escrever em prolog a árvore genealógica seguinte,
		e representar as relações familiares entre os indivíduos.
		\begin{figure}
			\centering
			\pgfuseimage{familia}
		\end{figure}		
	\end{example}
\end{frame}

\frame[label=showcode]{
\pgfdeclareimage[height=7.5cm]{showcode}{img/showcode}
\begin{figure}
\centering
\pgfuseimage{showcode}
\end{figure}}

\begin{frame}{Programa Heurístico}
	\begin{example}[Alice na Floresta do esquecimento]
		A Alice tinha má memória.
		Um dia entrou na floresta do Esquecimento e esquesceu-se do dia-da-semana.
		Os seus amigos \alert{Coelho} e \alert{Cuco} são visitantes frequentes da floresta. Estes dois
		são criaturas estranhas.\\
		O coelho \alert{mente} às \alert{Segundas}, \alert{Terças} e \alert{Quartas} e diz a \alert{verdade} no \alert{resto da Semana}.
		Por outro lado, o Cuco \alert{mente} às \alert{Quintas}, \alert{Sextas} e \alert{Sábados} \alert{dizendo a verdade no resto dos dias}.
	\end{example}
\end{frame}

\begin{frame}{Programa Heurístico}
	\begin{example}[Alice na Floresta do esquecimento]
		Um certo dia a Alice encontrou estes dois debaixo de uma árvore. Eles fizeram as seguintes declarações:
		\begin{itemize}
			\item Coelho: \alert{Ontem} foi um dos dias que eu \alert{menti};
			\item Cuco: \alert{Ontem} foi um dos dias que eu \alert{menti}.
		\end{itemize}
		
		A Alice foi capaz, usando estas declarações, de deduzir o dia-da-semana em que se encontrava.
	\end{example}
\end{frame}

\begin{frame}{Programa Heurístico}
	\begin{example}[Alice na Floresta do esquecimento]
		Com este exemplo pretende-se: 
		\begin{enumerate}
			\item Escrever uma Base de Conhecimento que descreva esta história;
			\item Escrever um predicado diadehoje/1 que lhe permita saber qual o dia-da-semana.
		\end{enumerate}
	\end{example}
\end{frame}

\againframe{showcode}

\begin{frame}{Extensões Bousi-Prolog}
	\begin{example}[Exemplos Bousi-Prolog]
		Em seguida serão apresentados os seguintes exemplos em Bousi-Prolog
		\begin{itemize}
			\item Programa de cálculo de idades;
			\item Programa de que emula um sistema de ``Information Retrieval'';
			\item Programa de escolha de apartamento inteligente.
		\end{itemize}
	\end{example}
\end{frame}

\againframe{showcode}

\section{E agora?}

\begin{frame}{O que pode ser feito?}
	\begin{itemize}[<+->]
		\item Estudar o sistema que esta a ser desenvolvido na universidade
		de Málaga \alert{FSQL} (Fuzzy SQL) e tentar integrar nos sistemas actuais
		(Neste site propões-se a interacção com SQL Server e Oracle).
		\item Estudar as funções Fuzzy disponibilizadas nos SGBDs actuais 
		(exemplo Soundex e Difference no SQL SERVER).
		\item Estudar os algoritmos de procura de pares em Fuzzy (Busca de informação repetida).
	\end{itemize}
\end{frame}

\begin{frame}{O que há para fazer ainda em Fuzzy?}
	\begin{itemize}[<+->]
		\item Implementar sistemas fuzzy com relações n-árias;
		\item Estudar a possibilidade de desenvolver sistemas com Conjuntos Analógicos;
		\item Modelação de um sistema deductivo de pesquisa de documentação inteligente.
	\end{itemize}
\end{frame}


\section{Sumário}

\begin{frame}{Sumário}
\begin{block}{Recapitulando\ldots}
	\begin{itemize}[<+->]
		\item Vimos motivação do estudo da lógica Fuzzy, bem como
		algumas vantagens da implementação desta;
		\item Estudamos os conceitos Fundamentais da Lógica Fuzzy;
		\item Resolvemos um exercício de Decisão recorrendo 
		à metodologia Mamdani;
		\item Apresentamos o sistema Bousi-Prolog, analizando para
		isso vários exemplos de programas em Prolog e Bousi-Prolog;
		\item Discutimos onde poderia ser usado o conceito na
		nossa corporação e melhorias que poderiam ser feitas a este. 
	\end{itemize}
\end{block}
\end{frame}

\begin{frame}{Dúvidas?}
\pgfdeclareimage[height=4.5cm]{duvidas}{img/duvidas}
		\begin{figure}
			\centering
			\pgfuseimage{duvidas}
		\end{figure}		
\end{frame}

%%%%%%%%%%%%%%%%%%%%%%%%%%%%%%%%%%%%%%%%%%%%%%%%%%%%%%%%%%%%%%%%%%%%%%%%%%%%%

%%%%%%%%%%%%%%%%%%%%%%%%%%%%%%%BIBLIOGRAFIA%%%%%%%%%%%%%%%%%%%%%%%%%%%%%%%%%%%%%%%%%%%%%%
\section*{Referências}

\begin{frame}[allowframebreaks]
  \frametitle<presentation>{Para ler depois}
    
  \begin{thebibliography}{10}
    
  \beamertemplatebookbibitems
  \bibitem{Tanaka1996}
    Kazuo Tanaka.
    \newblock {\em An Introduction to Fuzzy Logic for Practical Applications}.
    \newblock Springer, 1996.

  \bibitem{Shapiro1986}
    Shapiro.
    \newblock {\em The Art of Prolog}.
    \newblock MIT Press, 1986.
 
  \beamertemplatearticlebibitems
  % Followed by interesting articles. Keep the list short. 

  \bibitem{Zadeh1965}
    Zadeh.
    \newblock Fuzzy Sets.
    \newblock {\em Information and Control}, 8(3):338--353,
    1965.
  \end{thebibliography}
\end{frame}

\appendix
\section{\appendixname}
\frame{\tableofcontents}
\subsection{Material Adicional}

\frame[label=eqpertencadetail]{
  \frametitle{Definições Fuzzy --- Etiquetas e Conjuntos Fuzzy Detalhe}

$$
		\mu_X(x_0) = \left \{ \begin{array}{ll}
				$1$ & \textrm{sse $x \in \mathcal{C}$} \\
				$0$ & \textrm{sse $x \notin \mathcal{C}$} \\
				$0$ \le \mu_X(x_0) \le 1 & \textrm{sse $x \sim \in \mathcal{C}$} \\
			\end{array} \right . 
		$$	

    \hfill\hyperlink{eqpertenca}{\beamerreturnbutton{Voltar}}
}

\end{document}



\end{verbatim}

\item
  You can now run |pdflatex| or |latex| on the two files |main.beamer.tex| and |main.article.tex|.
\end{itemize}

The command |\setjobnamebeamerversion| tells the article version where to find the presentation version. This is necessary if you wish to include slides from the presentation version in an article as figures.

\begin{command}{\setjobnamebeamerversion\marg{filename without extension}}
  Tells the \beamer\ class where to find the presentation version of the current file.
\end{command}

An example of this workflow approach can be found in the |examples| subdirectory for files starting with |beamerexample2|.

\subsubsection{Including Slides from the Presentation Version in the Article Version}

If you use the package |beamerarticle|, the |\frame| command becomes available in |article| mode. By adjusting the frame template, you can ``mimic'' the appearance of frames typeset by \beamer\ in your articles. However, sometimes you may wish to insert ``the real thing'' into the |article| version, that is, a precise ``screenshot'' of a slide from the presentation. The commands introduced in the following help you do exactly this.

In order to include a slide from your presentation in your article version, you must do two things: First, you must place a normal \LaTeX\ label on the slide using the |\label| command. Since this command is overlay-specification-aware, you can also select specific slides of a frame. Also, by adding the option |label=|\meta{name} to a frame, a label \meta{name}|<|\meta{slide number}|>| is automatically added to each slide of the frame.

Once you have labeled a slide, you can use the following command in your article version to insert the slide into it:

\begin{command}{\includeslide\oarg{options}\marg{label name}}
  This command calls |\pgfimage| with the given \meta{options} for the file specified by
  \begin{quote}
    |\setjobnamebeamerversion|\meta{filename}
  \end{quote}
  Furthermore, the option |page=|\meta{page of label name} is passed to |\pgfimage|, where the \meta{page of label name} is read internally from the file \meta{filename}|.snm|.
  \example

\begin{verbatim}
\article
  \begin{figure}
    \begin{center}
      \includeslide[height=5cm]{slide1}
    \end{center}
    \caption{The first slide (height 5cm). Note the partly covered second item.}
  \end{figure}
  \begin{figure}
    \begin{center}
      \includeslide{slide2}
    \end{center}
    \caption{The second slide (original size). Now the second item is also shown.}
  \end{figure}
\end{verbatim}
\end{command}

The exact effect of passing the option |page=|\meta{page of label name} to the command |\pgfimage| is explained in the documentation of |pgf|. In essence, the following happens:
\begin{itemize}
\item
  For old versions of |pdflatex| and for any version of |latex| together with |dvips|, the |pgf| package will look for a file named
  \begin{quote}
    \meta{filename}|.page|\meta{page of label name}|.|\meta{extension}
  \end{quote}
  For each page of your |.pdf| or |.ps| file that is to be included in this way, you must create such a file by hand. For example, if the PostScript file of your presentation version is named |main.beamer.ps| and you wish to include the slides with page numbers 2 and~3, you must create (single page) files |main.beamer.page2.ps| and |main.beamer.page3.ps| ``by hand'' (or using some script). If these files cannot be found, |pgf| will complain.
\item
  For new versions of |pdflatex|, |pdflatex| also looks for the files according to the above naming scheme. However, if it fails to find them (because you have not produced them), it uses a special mechanism to directly extract the desired page from the presentation file |main.beamer.pdf|.
\end{itemize}


\subsection{Details on Modes}
\label{section-mode-details}

This subsection describes how modes work exactly and how you can use the |\mode| command to control what part of your text belongs to which mode.

When \beamer\ typesets your text, it is always in one of the following five modes:
\begin{itemize}
\item
  \declare{|beamer|} is the default mode.
\item
  \declare{|second|} is the mode used when a slide for an optional second screen is being typeset.
\item
  \declare{|handout|} is the mode for creating handouts.
\item
  \declare{|trans|} is the mode for creating transparencies.
\item
  \declare{|article|} is the mode when control has been transferred to another class, like |article.cls|. Note that the mode is also |article| if control is transferred to, say, |book.cls|.
\end{itemize}

In addition to these modes, \beamer\ recognizes the following names for modes sets:

\begin{itemize}
\item
  \declare{|all|} refers to all modes.
\item
  \declare{|presentation|} refers to the first four modes, that is, to all modes except for the |article| mode.
\end{itemize}

Depending on the current mode, you may wish to have certain text inserted only in that mode. For example, you might wish a certain frame or a certain table to be left out of your article version. In some situations, you can use the |\only| command for this purpose. However, the command |\mode|, which is described in the following, is much more powerful than |\only|.

The command actually comes in three ``flavors,'' which only slightly differ in syntax. The first, and simplest, is the version that takes one argument. It behaves essentially the same way as |\only|.

\begin{command}{\mode\sarg{mode specification}\marg{text}}
  Causes the \meta{text} to be inserted only for the specified modes. Recall that a \meta{mode specification} is just an overlay specification in which no slides are mentioned.

  The \meta{text} should not do anything fancy that involves mode switches or including other files. In particular, you should not put an |\include| command inside \meta{text}. Use the argument-free form below, instead.

  \example
\begin{verbatim}
\mode<article>{Extra detail mentioned only in the article version.}

\mode
<beamer| trans>
{\frame{\tableofcontents[currentsection]}}
\end{verbatim}
\end{command}

The second flavor of the |\mode| command takes no argument. ``No argument'' means that it is not followed by an opening brace, but any other symbol.

\begin{command}{\mode\sarg{mode specification}}
  In the specified mode, this command actually has no effect. The interesting part is the effect in the non-specified modes: In these modes, the command causes \TeX\ to enter a kind of ``gobbling'' state. It will now ignore all following lines until the next line that has a sole occurrence of one of the following commands: |\mode|, |\mode*|, |\begin{document}|, |\end{document}|. Even a comment on this line will make \TeX\ skip it. Note that the line with the special commands that make \TeX\ stop gobbling may not directly follow the line where the gobbling is started. Rather, there must either be one non-empty line before the special command or at least two empty lines.

  When \TeX\ encounters a single |\mode| command, it will execute this command. If the command is |\mode| command of the first flavor, \TeX\ will resume its ``gobbling'' state after having inserted (or not inserted) the argument of the |\mode| command. If the |\mode| command is of the second flavor, it takes over.

  Using this second flavor of |\mode| is less convenient than the first, but there are different reasons why you might need to use it:
  \begin{itemize}
  \item
    The line-wise gobbling is much faster than the gobble of the third flavor, explained below.
  \item
    The first flavor reads its argument completely. This means, it cannot contain any verbatim text that contains unbalanced braces.
  \item
    The first flavor cannot cope with arguments that contain |\include|.
  \item
    If the text mainly belongs to one mode with only small amounts of text from another mode inserted, this second flavor is nice to use.
  \end{itemize}

  \emph{Note:} When searching line-wise for a |\mode| command to shake it out of its gobbling state, \TeX\ will not recognize a |\mode| command if a mode specification follows on the same line. Thus, such a specification must be given on the next line.

  \emph{Note:} When a \TeX\ file ends, \TeX\ must not be in the gobbling state. Switch this state off using |\mode| on one line and |<all>| on the next.

  \example
\begin{verbatim}
\mode<article>

This text is typeset only in |article| mode.
\verb!verbatim text is ok {!

\mode
<presentation>
{ % this text is inserted only in presentation mode
\frame{\tableofcontents[currentsection]}}

Here we are back to article mode stuff. This text
is not inserted in presentation mode

\mode
<presentation>

This text is only inserted in presentation mode.
\end{verbatim}
\end{command}

The last flavor of the mode command behaves quite differently.

\begin{command}{\mode\declare{|*|}}
  The effect of this mode is to ignore all text outside frames in the |presentation| modes. In |article| mode it has no effect.

  This mode should only be entered outside of frames. Once entered, if the current mode is a |presentation| mode, \TeX\ will enter a gobbling state similar to the gobbling state of the second ``flavor'' of the |\mode| command. The difference is that the text is now read token-wise, not line-wise. The text is gobbled token by token until one of the following tokens is found: |\mode|, |\frame|, |\againframe|, |\part|, |\section|, |\subsection|, |\appendix|, |\note|, |\begin{frame}|, and |\end{document}| (the last two are really tokens, but they are recognized anyway).

  Once one of these commands is encountered, the gobbling stops and the command is executed. However, all of these commands restore the mode that was in effect when they started. Thus, once the command is finished, \TeX\ returns to its gobbling.

  Normally, |\mode*| is exactly what you want \TeX\ to do outside of frames: ignore everything except for the above-mentioned commands outside frames in |presentation| mode. However, there are  cases in which you have to use the second flavor of the |\mode| command instead: If you have verbatim text that contains one of the commands, if you have very long text outside frames, or if you wish some text outside a frame (like a definition) to be executed also in |presentation| mode.

  The class option |ignorenonframetext| will switch on |\mode*| at the beginning of the document.

  \example
\begin{verbatim}
\begin{document}
\mode*

This text is not shown in the presentation.

\begin{frame}
  This text is shown both in article and presentation mode.
\end{frame}

this text is not shown in the presentation again.

\section{This command also has effect in presentation mode}

Back to article stuff again.

\frame<presentation>
{ this frame is shown only in the presentation. }
\end{document}
\end{verbatim}

  \example The following example shows how you can include other files in a main file. The contents of a |main.tex|:

\begin{verbatim}
\documentclass[ignorenonframetext]{beamer}
\begin{document}
This is star mode stuff.

Let's include files:
\mode<all>
\include{a}
\include{b}
\mode*

Back to star mode
\end{document}
\end{verbatim}

  And |a.tex| (and likewise |b.tex|):

\begin{verbatim}
\mode*
\section{First section}
Extra text in article version.
\begin{frame}
  Some text.
\end{frame}
\mode<all>
\end{verbatim}
\end{command}
