% $Header: /home/vedranm/bitbucket/beamer/solutions/conference-talks/conference-ornate-20min.en.tex,v 90e850259b8b 2007/01/28 20:48:30 tantau $

\documentclass[portuges]{beamer}

\mode<presentation>
{
  \usetheme{Warsaw}
  \setbeamercovered{transparent}
}

\usepackage{babel}

\usepackage[utf8x]{inputenc}

\usepackage{times}
\usepackage[T1]{fontenc}
\usepackage{amsmath}
\usepackage{latexsym}
%\usepackage{amstex}

\title[Fuzzy \& SIAD] %
{Lógica Fuzzy aplicada a \\Sistemas de Informação de Apoio à Decisão}


\author[António Sérgio Matos da Silva] %
{António Sérgio Matos da Silva \\ \texttt{an.silva@logica.com}}

\institute[Logica] % 
{
  Telecommunication Business\\
  Logica
}

\date[Reunião Mensal Zon] % 
{\today\\Reunião Mensal Zon}

\subject{Theoretical Computer Science}


% Delete this, if you do not want the table of contents to pop up at
% the beginning of each subsection:
\AtBeginSubsection[]
{
  \begin{frame}<beamer>{Programa}
    \tableofcontents[currentsection,currentsubsection]
  \end{frame}
}

\pgfdeclareimage[height=0.5cm]{logo}{img/logo}
\logo{\pgfuseimage{logo}}

%\beamerdefaultoverlayspecification{<+->}

%%%%%%%%%%%%%%%%%%%%%%%%%%%%%%%%%%%%MY SPECS
\newtheorem{convention}[theorem]{Convenção}
\newtheorem{proposition}[theorem]{Proposição}

\begin{document}

\begin{frame}
  \titlepage
\end{frame}

\begin{frame}
\begin{quote}
``À medida que a complexidade aumenta, 
as declarações precisas perdem relevância e as declarações relevantes perdem precisão.'' 
\end{quote}
\begin{quote}
Lofti Zadeh
\end{quote}
\end{frame}

\begin{frame}{Programa}
  \tableofcontents[pausesections]
\end{frame}
%%%%%%%%%%%%%%%%%%%%%%%%%%%%%%APRESENTACAO%%%%%%%%%%%%%%%%%%%%%%%%%%%%%%%%

\section{Breve contextualização Teórica}
\subsection{Motivação}

\begin{frame}{Como ``Raciocina'' um Computador Tradicional?}
\pgfdeclareimage[width=.6\textwidth]{pcthink}{img/pcthink}
\begin{figure}
\centering
\pgfuseimage{pcthink}
\caption{Funcionamento de um PC}
\end{figure}
\end{frame}


\begin{frame}{Pensamento Lógico Humano}
\pgfdeclareimage[width=.6\textwidth]{larrythink}{img/larrythink}
\begin{figure}
\centering
\pgfuseimage{larrythink}
\caption{Funcionamento do cérebro de Larry Wall}
\end{figure}
\end{frame}

\begin{frame}{Motivação}
\begin{itemize}[<+->]
	\item Tradicionalmente, o comportamento dos sistemas é realizado
		pela representação da lógica de Boole, aceitando apenas dois resultados
		\alert{zero ou um}, \alert{verdadeiro ou falso}, tudo ou nada.
	\item No entanto, em muitas situações não relevamos esta dicotomia comportamental.
		\begin{itemize}
			\item Apesar de à partida, uma pessoa com $1.75\ m$ ser alta, esta não é assim
			tão alta;
			\item Quando alguém nos diz que nos ama, \alert{não sabemos o quanto} esta nos ama;
			\item Nem sempre precisamos de obter um resultado baseado numa \alert{certeza},
			ficamos satisfeitos apenas com um certo grau de \alert{confiança}.
		\end{itemize}		 
\end{itemize}

\end{frame}

\begin{frame}{Analogia}
	\pgfdeclareimage[height=2.5cm]{rightcurve}{img/rightcurve}
	\begin{example}
			\begin{figure}
			\centering
			\pgfuseimage{rightcurve}
			\end{figure}

			Pretende-se comparar os dois mecanismos
			para efectuar a curva à direita.		 	
	\end{example}
\end{frame}
\begin{frame}{Analogia}
\begin{columns}[T]
  		\begin{column}{5cm}
		    \begin{block}{Booleana}<1->  				
				\begin{enumerate}[<+->]
					\item Pressionar o travão com uma força de $20\ Newtons$.
					\item Inclinar o volante $15\ graus$ para a Direita
					\item Colocar o volante na posição inicial ($0\ graus$).
				\end{enumerate}		
			\end{block}
		  \end{column}
		  \begin{column}{5cm}
		    \begin{block}{Humana}<4->
  			\begin{enumerate}[<+->]
					\item Reduza a velocidade.

					\item Vire \alert{um pouco} para a direita
					\item Vire \alert{mais um pouco} para a direita
					\item Siga em frente.
				\end{enumerate}		
		    \end{block}

		  \end{column}
		\end{columns}

\end{frame}


\subsection{História e Uso da Lógica Fuzzy}

\begin{frame}{Resumo Histórico\ldots}

	\pgfdeclareimage[height=3cm]{lukasiewicz}{img/Lukasiewicz}
   \begin{columns}
	\begin{column}{7cm}
	\begin{itemize}[<+->]
		\item[1930] Jan Lukasiewz propôs o estudo de termos qualitativos do tipo alto, velho e quente 
		e propôs a idéia da utilização de um intervalo de valores entre $0$ e $1$ para descrever a veracidade
		de uma dada afirmação;
		\item[1937] Max Black definiu o primeiro conjunto fuzzy e descreveu algumas idéias básicas de operações com estes.
	\end{itemize}
	\end{column}
	\begin{column}{4cm}
		\begin{figure}
			\centering
			\pgfuseimage{lukasiewicz}
			\caption{Jan Lukasiewicz (1878--1956)}
		\end{figure}
	\end{column}	
   \end{columns}
\end{frame}

\begin{frame}{Resumo Histórico\ldots}
\pgfdeclareimage[height=3cm]{zadeh}{img/zadeh}
   \begin{columns}
	\begin{column}{7cm}
	\begin{itemize}[<+->]
		\item[1965] Lofti Zadeh publicou o artigo Fuzzy Sets que ficou
		conhecido como a origem da Lógica Fuzzy. Zadeh é conhecido como
		o ``mestre'' da Lógica Fuzzy.
	\end{itemize}
	\end{column}
	\begin{column}{4cm}
		\begin{figure}
			\centering
			\pgfuseimage{zadeh}
			\caption{Lofti Zadeh (1921--Actualidade)}
		\end{figure}
	\end{column}	
   \end{columns}
\end{frame}

\begin{frame}{Uso da Lógica Fuzzy}
\begin{itemize}
	\item[1970] Primeira aplicação da Lógica Fuzzy na engenharia de controlo;
	\item[1975] Introdução da Lógica Fuzzy no Japão;
        \item[1985] Ampla utilização no Japão;
        \item[1990] Ampla utilização na Europa;
        \item[1995] Ampla utilização nos EUA;
        \item[1996] 1100 aplicações com Lógica Fuzzy publicadas; 
	\item[2000] Aplicada a finanças e controle multi-variável.
\end{itemize}
\end{frame}

\begin{frame}{Porquê usar Fuzzy?}
\begin{block}{Vantagens da Lógica Fuzzy}
\begin{itemize}
	\item<1-> Robusta porque não requer entradas precisas;
	\item<2-> Facilmente modificável pois é baseada em regras;
	\item<3-> Evita o formalismo matemático para sistemas não lineares;
	\item<3-> Solução rápida e barata para sistemas complexos não lineares;
	\item<4-> Implementável em microprocessadores.
\end{itemize}
\end{block}
\end{frame}


\subsection{Fundamentos da Lógica Fuzzy}

\begin{frame}{Conceito de Difusidade}
	\pgfdeclareimage[height=3cm]{arcond}{img/arcond}  	
	\begin{figure}
			\centering
			\pgfuseimage{arcond}
	\end{figure}
		
	\begin{example}[O caso do ar condicionado]
  		No edifício Pinta o ar condicionado encontra-se constantemente
		avariado. Pretende-se desenvolver uma lógica que pretende 
		avaliar se uma dada temperatura é \alert{confortável} ou não.
	\end{example}
\end{frame}

\begin{frame}{Conceito de Difusidade}
	\pgfdeclareimage[width=.6\textwidth]{fuzzy1}{img/fuzzy1}	
	\begin{quote}
		Numa primeira análise vem o José e diz que a temperatura ideal
		para ele é de exactamente $20º\ C$.
	\end{quote}
	\begin{figure}
			\centering
			\pgfuseimage{fuzzy1}
			\caption{Função de Verdade de uma Lógica tipicamente Booleana.}
	\end{figure}
\end{frame}

\begin{frame}{Conceito de Difusidade}
	\pgfdeclareimage[width=.5\textwidth]{fuzzy2}{img/fuzzy2}	
	\begin{quote}
		No entanto o Miguel discorda com valor estipulado anteriormente. ``Vamos
		lá ver! Se a temperatura for 18 ou 22 não deixa de ser bom também!'' 
	\end{quote}
	\begin{figure}
			\centering
			\pgfuseimage{fuzzy2}
			\caption{Função de Verdade de uma Lógica na Cabeça do ``Miguel''.}
	\end{figure}
\end{frame}

\begin{frame}{Conceito de Difusidade}
	\pgfdeclareimage[width=.4\textwidth]{fuzzy3}{img/fuzzy3}	
	\begin{quote}
		No dia seguinte já com o ar condicionado a funcionar e uma
		temperatura a funcionar o Miguel pergunta ao Pedro.\\
		``Bom dia! Está bom tempo hoje!''\\
		Ao que este responde:\\
		``Sim está. \alert{Nem muito quente nem muito frio}\ldots''  
	\end{quote}
	\begin{figure}
			\centering
			\pgfuseimage{fuzzy3}
			\caption{Função de Verdade de uma Lógica Fuzzy.}
	\end{figure}
\end{frame}

\begin{frame}{Conceito de Difusidade}
	\pgfdeclareimage[width=.6\textwidth]{fuzzy3_1}{img/fuzzy3_1}	
	\begin{quote}
		Na verdade o que o Pedro pensou\ldots
	\end{quote}
	\begin{figure}
			\centering
			\pgfuseimage{fuzzy3_1}
			\caption{Função de Verdade de uma Lógica Fuzzy representando mais do que uma valoração ao mesmo tempo.}
	\end{figure}
\end{frame}

\begin{frame}{Definições Fuzzy --- Etiqueta linguística}
	\begin{definition}
		A descrição de inúmeras situações concretas referentes a uma dada variável faz-se por intermédio de \alert{etiquetas
		linguísticas}. Estas representam o carácter \alert{qualitativo} de todas as possibilidades de uma dada variável.
	\end{definition}
	
	\begin{exampleblock}{Etiquetas Bivalentes}<2->
		\begin{itemize}
			\item Lento;
			\item Rápido.
		\end{itemize}	
	\end{exampleblock}
\end{frame}

\begin{frame}{Definições Fuzzy --- Exemplo de Etiquetas Linguísticas}
	\begin{exampleblock}{Etiquetas Trivalentes}
		\begin{itemize}
			\item Baixo;
			\item Médio;
			\item Alto.
		\end{itemize}	
	\end{exampleblock}
	
	\begin{exampleblock}{Etiquetas Multivalentes}
		\begin{itemize}
			\item Muito Frio
			\item Frio;
			\item Moderado;
			\item Quente;
			\item Muito Quente.
		\end{itemize}	
	\end{exampleblock}
\end{frame}

\begin{frame}{Definições Fuzzy --- Variável Linguística}
		\pgfdeclareimage[width=.6\textwidth]{continuousfuzzy}{img/fuzzy3}	
	\begin{definition}<1->
		Uma variável linguística pode ser contínua ou discreta.
		\end{definition}
	\begin{example}<2->[Temperatura]	
	\begin{figure}
			\centering
			\pgfuseimage{continuousfuzzy}
			\caption{Exemplo de Variável Linguística contínua.}
	\end{figure}
	\end{example}
\end{frame}

\begin{frame}
	\pgfdeclareimage[width=.6\textwidth]{discretefuzzy}{img/fuzzydiscrete}
	\begin{example}[Rodas de um camião]
	\begin{figure}
			\centering
			\pgfuseimage{discretefuzzy}
			\caption{Exemplo de Variável Linguística Discreta.}
	\end{figure}	
	\end{example}
\end{frame}

\begin{frame}{Definições Fuzzy --- Conjuntos Fuzzy no Mundo Real}
	\pgfdeclareimage[width=.4\textwidth]{realworldfuzzy}{img/realworldfuzzy}
	\begin{convention}
		Geralmente os Sistemas de Informação baseados num sistema de decisão Fuzzy
		usam a seguinte simplificação das etiquetas. A partir de agora usaremos também
		esta notação.
	\end{convention}

	\begin{example}[Conjunto Fuzzy no Mundo Real]
	\begin{figure}
			\centering
			\pgfuseimage{realworldfuzzy}
			\caption{Exemplo de Variável Linguística Contínua do Mundo Real.}
	\end{figure}	
	\end{example}
\end{frame}

\frame[label=eqpertenca]
{
  \frametitle{Definições Fuzzy --- Etiquetas e Conjuntos Fuzzy}
	\pgfdeclareimage[width=.4\textwidth]{pertenca}{img/pertenca}  	
	\begin{definition}
		Seja $X$ uma Etiqueta Linguística. Esta é representada por um Conjunto Fuzzy $\mathcal{C}$ descrito pela
		função de pertença $\mu_X(x_0)$.
	\end{definition}
		
	\begin{figure}
			\centering
			\pgfuseimage{pertenca}
			\caption{Etiqueta Linguística e respectiva Função Pertença.}
	\end{figure}		

    \hyperlink{eqpertencadetail}{\beamergotobutton{Equação Detalhe}}
}

\begin{frame}{Definições Fuzzy --- Suporte}
\pgfdeclareimage[width=.4\textwidth]{support}{img/support}  	
\begin{definition}
		Seja $X$ uma Etiqueta Linguística. Designa-se por Suporte de $X$ ($\mathcal{S}_X$)
		a zona em que a sua função pertença não é nula. Ou seja: $\{\mu_X(x_0) \ \vert \ \mu_X(x_0) \neq 0 \land x_0 \in [0,1] \}$
	\end{definition}

	\begin{figure}
			\centering
			\pgfuseimage{support}
			\caption{Suporte de uma Etiqueta Linguística}
	\end{figure}
\end{frame}

\begin{frame}{Definições Fuzzy --- Núcleo}
\pgfdeclareimage[width=.4\textwidth]{nucleo}{img/nucleo}  	
\begin{definition}
		Seja $X$ uma Etiqueta Linguística. Designa-se por Núcleo de $X$ ($\mathcal{N}_X$)
		a zona em que a sua função pertença é máxima. Ou seja: $\{\mu_X(x_0) \ \vert \ \mu_X(x_0) = 1 \land x_0 \in [0,1] \}$
	\end{definition}

	\begin{figure}
			\centering
			\pgfuseimage{nucleo}
			\caption{Núcleo de uma Etiqueta Linguística}
	\end{figure}
\end{frame}

\begin{frame}{Definições Fuzzy --- Variável Linguística Normada}  	
\pgfdeclareimage[width=.5\textwidth]{notnormada}{img/notnormada}  	
\begin{definition}
		Uma Variável Línguística diz-se Normada se esta gera um discurso limitado pelo 
		conjunto $[-1,1]$
	\end{definition}
		\begin{figure}
			\centering
			\pgfuseimage{notnormada}
			\caption{Variável não normada.}
	\end{figure}
\end{frame}

\begin{frame}{Definições Fuzzy --- Variável Linguística Normada Exemplo}  	
\pgfdeclareimage[width=.6\textwidth]{normada}{img/normada}  	
	\begin{example}[Variável Linguística Normada]		
	\begin{figure}
			\centering
			\pgfuseimage{normada}
			\caption{Variável normada à escala 1/120.}
	\end{figure}
	\end{example}
\end{frame}

\begin{frame}{Operações Com Conjuntos Fuzzy}
	\begin{proposition}
		Seja $\mho$ o Universo de uma Variável Linguística
		e sejam $\mathcal{A}$ e $\mathcal{B}$ dois Conjuntos Fuzzy 
		definidos por:
		$$
			\begin{array}{ccc}
				\mathcal{A} & = & \{ (x,\mu_\mathcal{A}(x))\ \vert\ x \in \mho \land \mu_\mathcal{A}(x) \in [0,1] \} \\
				\mathcal{B} & = & \{ (x,\mu_\mathcal{B}(x))\ \vert\ x \in \mho \land \mu_\mathcal{B}(x) \in [0,1] \}
			\end{array}
		$$   
	\end{proposition}
\end{frame}

\begin{frame}{Operações Com Conjuntos Fuzzy --- União}
	\pgfdeclareimage[width=.4\textwidth]{orvenn}{img/orvenn}  	

	\begin{definition}[União]
	A união entre $\mathcal{A}$ e $\mathcal{B}$ é definida por:
	$$	
	\mathcal{A} \cup \mathcal{B} = \{ ( x , max(\mu_\mathcal{A}(x),\mu_\mathcal{B}(x)))\ \vert\ x \in \mho \} 
	$$
	\end{definition}	
\begin{figure}
			\centering
			\pgfuseimage{orvenn}
			\caption{Diagrama de Venn da União de dois Conjuntos}
	\end{figure}	
\end{frame}

\begin{frame}{Operações Com Conjuntos Fuzzy --- Intersecção}
	\pgfdeclareimage[width=.4\textwidth]{andvenn}{img/andvenn}  	

	\begin{definition}[Intersecção]
	A Intersecção entre $\mathcal{A}$ e $\mathcal{B}$ é definida por:
	$$	
	\mathcal{A} \cap \mathcal{B} = \{ ( x , min(\mu_\mathcal{A}(x),\mu_\mathcal{B}(x)))\ \vert\ x \in \mho \} 
	$$
	\end{definition}	
\begin{figure}
			\centering
			\pgfuseimage{andvenn}
			\caption{Diagrama de Venn da Intersecção de dois Conjuntos}
	\end{figure}	
\end{frame}

\begin{frame}{Operações Com Conjuntos Fuzzy --- Complemento}
	\pgfdeclareimage[width=.25\textwidth]{notvenn}{img/notvenn}  	

	\begin{definition}[Complemento]
	O Complemento de $\mathcal{A}$ é definido por:
	$$	
	\lnot\mathcal{A} = \{ ( x , \mu_{\lnot\mathcal{A}}(x) )\ \vert\ x \in \mho \land \mu_{\lnot\mathcal{A}}(x) = 1 - \mu_{\mathcal{A}}(x) \} 
	$$
	\end{definition}	
\begin{figure}
			\centering
			\pgfuseimage{notvenn}
			\caption{Diagrama de Venn do Complemento de um Conjunto}
	\end{figure}	
\end{frame}

\begin{frame}{Modificadores Linguísticos}
	\begin{definition}
		Seja $\mathcal{A}$ um conjunto Fuzzy intervalar com a função de pertinência $\mu_\mathcal{A}{x}$.
		Então o \emph{Modificador Línguístico} de $\mathcal{A}$ é uma função intervalar $\mathcal{M}$ definida por:
		$$\mathcal{M}:\mathcal{I}[0,1] \to \mathcal{I}[0,1] $$
		que age na função pertinência $\mu_{\mathcal{I}\mathcal{A}}{x}$ transformando-a em $\mu_{m\mathcal{I}\mathcal{A}}{x}$ onde:
		$$\mu_{m\mathcal{I}\mathcal{A}}(x) = \mathcal{M}(\mu_{\mathcal{I}\mathcal{A}}(x))$$
	\end{definition}
\end{frame}

\begin{frame}{Modificadores Linguísticos --- Very}
	\pgfdeclareimage[height=3cm]{muito}{img/muito}  		
	\begin{definition}[Very]
		O modificador \alert{Very} (muito) define-se por:
		$$\mu_{v\mathcal{A}}{x} = \mu_{v\mathcal{A}}^2{x}$$ 
	\end{definition}
	\begin{figure}
			\centering
			\pgfuseimage{muito}
			\caption{Representação Gráfica do Modificador Very}
	\end{figure}	
\end{frame}

\begin{frame}{Modificadores Linguísticos --- Somewhat}
	\pgfdeclareimage[height=3cm]{pouco}{img/pouco}  		
	\begin{definition}[Somewhat]
		O modificador \alert{Somewhat} (pouco) define-se por:
		$$\mu_{v\mathcal{A}}{x} = \sqrt[2]{\mu_{v\mathcal{A}}{x}}$$ 
	\end{definition}
	\begin{figure}
			\centering
			\pgfuseimage{pouco}
			\caption{Representação Gráfica do Modificador Somewhat}
	\end{figure}	
\end{frame}

\begin{frame}{Modificadores Linguísticos --- Above}
	\pgfdeclareimage[height=3cm]{acima}{img/acima}  		
	\begin{definition}[Above]
		O modificador \alert{Above} (acima) define-se por:
		$$\mu_{v\mathcal{A}}{x} =\mu_{v\mathcal{A}}{x} - \delta$$ 
	\end{definition}
	\begin{figure}
			\centering
			\pgfuseimage{acima}
			\caption{Representação Gráfica do Modificador Above}
	\end{figure}	
\end{frame}

\begin{frame}{Modificadores Linguísticos --- Below}
	\pgfdeclareimage[height=3cm]{abaixo}{img/abaixo}  		
	\begin{definition}[Below]
		O modificador \alert{Below} (abaixo) define-se por:
		$$\mu_{v\mathcal{A}}{x} = \mu_{v\mathcal{A}}{x} + \delta$$ 
	\end{definition}
	\begin{figure}
			\centering
			\pgfuseimage{abaixo}
			\caption{Representação Gráfica do Modificador Below}
	\end{figure}	
\end{frame}

\begin{frame}{Modificadores Linguísticos --- Not}
	\pgfdeclareimage[height=3cm]{nao}{img/nao}  		
	\begin{definition}[Not]
		O modificador \alert{Not} (não) define-se por:
		$$\mu_{v\mathcal{A}}{x} = 1 - \mu_{v\mathcal{A}}{x}$$ 
	\end{definition}
	\begin{figure}
			\centering
			\pgfuseimage{nao}
			\caption{Representação Gráfica do Modificador Not}
	\end{figure}	
\end{frame}

\begin{frame}{Modificadores Linguísticos --- Not Very}
	\pgfdeclareimage[height=3cm]{naomuito}{img/naomuito}  		
	\begin{definition}[Not Very]
		O modificador \alert{Not Very} (não muito) define-se por:
		$$\mu_{v\mathcal{A}}{x} = 1 - \mu_{v\mathcal{A}}^2{x}$$ 
	\end{definition}
	\begin{figure}
			\centering
			\pgfuseimage{naomuito}
			\caption{Representação Gráfica do Modificador Not Very}
	\end{figure}	
\end{frame}

\begin{frame}{Inferência}
\begin{block}{Interferência}
\begin{itemize}[<+->]
\item
Para fazer deduções com conjuntos difusos utilizam-se \alert{regras de inferência},
		formatando afirmações condicionais como implicações do tipo ``if-then'';
\item O antecedente
		diz respeito às ``condições lógicas'' impostas sobre essa variável linguística;
\item O consequente
		diz respeito às ``acções'' decorrentes dessas condições na variável de saída;
\item No controlo difuso costumam haver múltiplas regras de inferência, de acordo com a natureza 
dos estados medidos no processo.
\end{itemize}
\end{block}
\end{frame}

\begin{frame}{Inferência}
\begin{example}[Base de Regras]
$$
	\begin{array}{ccccc}
	if & (antecedente_1) & then & (consequente_1) & or \\
	if & (antecedente_2) & then & (consequente_2) & or \\
	 &  & ... &  & \\
	if & (antecedente_n) & then & (consequente_n) &
	\end{array}
$$
\end{example}
\end{frame}

\begin{frame}{Pausa\ldots}
\pgfdeclareimage[width=0.9\textwidth]{kitkat}{img/kit_kat}  	
\begin{figure}
			\centering
			\pgfuseimage{kitkat}
	\end{figure}		
\end{frame}

\section{Estado da Arte}
\subsection{Estilo Mamdami}

\begin{frame}{História do Estilo Mamdami}
\pgfdeclareimage[width=2.5cm]{mamdani}{img/mamdani}
\begin{block}{Estilo Mamdani}
	\begin{columns}	
	\begin{column}{7cm}
	O estilo de inferência Mandani foi criado pelo professor Mandani da Universidade de Londres
	em 1975. O seu principal objectivo era desenvolver sistemas Fuzzy, baseando-se em regras de conjuntos Fuzzy 
	com o intuito de representar experiências da vida real.
	\end{column}
	\begin{column}{3cm}
		\begin{figure}
			\centering
			\pgfuseimage{mamdani}
			\caption{Ebrahim Mamdani (1943--2010)}
		\end{figure}		
	\end{column}
	\end{columns}
\end{block}
\end{frame}

\begin{frame}{Estilo Mamdani}
	\begin{definition}{Estilo Mamdani}
		O processo de raciocínio do \alert{Estilo Mamdani} é implementado seguindo as quatro etapas seguintes:
		\begin{enumerate}[<+->]
			\item Fuzzyficação;
			\item Avaliação das Regras Fuzzy;
			\item Agregação das Regras Fuzzy;
			\item Defuzzyficação.
		\end{enumerate}
	\end{definition}
\end{frame}

\begin{frame}{Exercício}
\begin{example}[Análise de Risco]
	Considere a análise de riscos num projecto. 
	Pretende-se estabelecer, sendo conhecidos um valor $x$ de recurso 
	monetário e um número $y$ de funcionários para trabalhar no mesmo,
	qual o risco $z$ nesse projecto.
\end{example}
\end{frame}

\begin{frame}{Variáveis de Entrada}
	\begin{columns}
	\begin{column}{5cm}
	\begin{tabular}{c|c} 
		\hline
		\multicolumn{2}{c}{{\textbf{Fundos do projecto($x$)}}}\\ 
		\hline
		Valor linguístico & Notação \\ \hline
		Inadequado & A1 \\
		Razoável & A2 \\ 
		Adequado & A3 \\ \hline
	\end{tabular}
	\end{column}
	\begin{column}{5cm}
	\begin{tabular}{c|c} 
		\hline
		\multicolumn{2}{c}{{\textbf{Funcionários do Projecto($y$)}}}\\ 
		\hline
		Valor linguístico & Notação \\ \hline
		Pequeno & B1 \\ 
		Grande & B2 \\ \hline
	\end{tabular}
	\end{column}
	\end{columns}
\end{frame}

\begin{frame}{Variáveis de Saída}
\begin{tabular}{c|c} 
		\hline
		\multicolumn{2}{c}{{\textbf{Risco do Projecto($z$)}}}\\ 
		\hline
		Valor linguístico & Notação \\ \hline
		Baixo & C1 \\
		Normal & C2 \\ 
		Alto & C3 \\ \hline
	\end{tabular}
\end{frame}

\begin{frame}{Fuzzyficação}
	\begin{example}[Entradas Crisp]
		Sejam $x$ e $y$ duas entradas crisp representando os conjuntos Fuzzy $X$ (Fundos do Projecto)
		e $Y$ (Funcionários do Projecto respectivamente). Então aplicando as entradas as conjuntos Fuzzy obtemos o valor
		das funções de pertença.
	\end{example}

	
\end{frame}

\begin{frame}{Fuzzyficacao da Variável referente aos Fundos do Projecto}
\pgfdeclareimage[height=4cm]{fuzzyfic1}{img/fuzzyfic1}
	\begin{figure}
			\centering
			\pgfuseimage{fuzzyfic1}
			\caption{Fuzzyficacao da Variável referente aos Fundos do Projecto}
		\end{figure}		
\end{frame}

\begin{frame}{Fuzzyficacao da Variável referente aos Funcionários do Projecto}
\pgfdeclareimage[height=4cm]{fuzzyfic2}{img/fuzzyfic2}
	\begin{figure}
			\centering
			\pgfuseimage{fuzzyfic2}
			\caption{Fuzzyficacao da Variável referente aos Funcionários do Projecto}
		\end{figure}		
\end{frame}

\begin{frame}{Variáveis de Entrada Fuzzificadas}
	\begin{columns}
	\begin{column}{5cm}
	\begin{tabular}{c|c} 
		\hline
		\multicolumn{2}{c}{{\textbf{Fundos do projecto ($x$)}}}\\ 
		\hline
		Etiqueta & Valor \\ \hline
		 A1 & $0.5$ \\
		A2  & $0.2$ \\ 
		A3 & $0$ \\ \hline
	\end{tabular}
	\end{column}
	\begin{column}{5cm}
	\begin{tabular}{c|c} 
		\hline
		\multicolumn{2}{c}{{\textbf{Funcionários do Projecto ($y$)}}}\\ 
		\hline
		Etiqueta & Valor \\ \hline
		 B1 & $0.1$ \\ 
		 B2 & $0.7$ \\ \hline
	\end{tabular}
	\end{column}
	\end{columns}
\end{frame}

\begin{frame}{Avaliação das Regras Fuzzy}
	\begin{example}[Avaliação das Regras Fuzzy]
		Com base nos graus de pertinência e nas correlações entre as variáveis linguísticas, têm-se as regras.
		\begin{enumerate}[<+->]
			\item \alert{IF} (($x$ is $A3 (0)$) \alert{OR} ($y$ is $B1 (0.1)$)) \alert{THEN} ($z$ is $C1 (0.1)$)
			\item IF (($x$ is $A2 (0.2)$) \alert{AND} ($y$ is $B2 (0.7)$)) THEN ($z$ is $C2 (0.2)$)
			\item IF ($x$ is $A1 (0.5)$ THEN ($z$ is $C3 (0.5)$))
		\end{enumerate}
	\end{example}
\end{frame}

\begin{frame}{Agregação das Regras Fuzzy}
	\pgfdeclareimage[height=4cm]{agrega}{img/agrega}
	\begin{figure}
			\centering
			\pgfuseimage{agrega}
			\caption{Conjunto Fuzzy Resultante do Processo de Agregação das Regras Fuzzy}
		\end{figure}		
\end{frame}

\begin{frame}{Defuzzyficação}
	\begin{definition}[Defuzzyficação]
		O método de defuzzyficação mais comum é a técnica do centróide, que 
		obtém o ponto onde uma linha vertical divide ao meio um conjunto agregado.
		A equação que descrever o cálculo da centróide é a seguinte $\mathcal{COG}$:
		$$
			\mathcal{COG} = \frac{\sum_{x=a}^{b}\mu(x) \cdot x}{\sum_{x=a}^{b}\mu(x)}
		$$
	\end{definition}
\end{frame}

\begin{frame}{Defuzzyficação --- Exemplo}
\begin{example}
Considerando o conjunto Fuzzy anterior, o resultado numérico obtido com a aplicação técnica do centróide $\mathcal{COG}$ é dado
por (considerando intervalos percentuais de $10\%$, variando de $0\%$ a $100\%$):
		$$
			\mathcal{COG} = \frac{ \begin{array}{c}
						(0 + 10 + 20) \cdot 0.1 + \\
						(30 + 40 +50) \cdot 0.2 + \\
						(60 + 70 + 80 + 90 + 100)\cdot 0.5 
						\end{array}}
					{\begin{array}{c}
						0.1 + 0.1 + 0.1 + \\
						0.2 + 0.2 + 0.2 + \\
						0.5 + 0.5 + 0.5 + 0.5
					 \end{array}} = 67.4
		$$
Assim tem-se que o risco do projecto em questão é de $67.4\%$.
\end{example}
\end{frame}

\subsection{Bousi-Prolog}

\begin{frame}{Breve História\ldots}
\begin{itemize}[<+->]
	\item[-] Os matemáticos descobriram que apesar da lógica 
	de primeira ordem não ser automaticamente dedutível,
	existem subconjuntos que o são;
	\item[1965] Robinson definiu a dedução automática;
	\item[1969] Green Implementou um sistema de Resolução em Lisp
	\item[1970] Kowalsky começa a usar as Cláusulas de Horn (subconjunto
	da lógica da 1ª ordem) para ``provas automáticas''.
	
\end{itemize}
\end{frame}

\begin{frame}{Breve História\ldots}
	\begin{itemize}[<+->]
		\item[1972] Um grupo de investigadores da Universidade de Marselha desenvolveu 
		um sistema de resolução para as Cláusulas de Horn;
		\item[1980-] O governo Japonês investiu no projecto designado por quinta geração que
		teve como resultado grandes contribuições para a computação lógica;
		\item[2008] Julián-Iranzo, Rubio-Manzano e Gallardo Casero 
	propuseram uma extensão à máquina de inferência Prolog,
	utilizando lógica Fuzzy, para que existissem ``respostas
	mais flexíveis às perguntas''. Para isso foi implementado
	o sistema Bousi~Prolog e continua em desenvolvimento na
	Universidad de Castilla-La Mancha.	
	\end{itemize}
\end{frame}

\begin{frame}{Porquê Prolog?}
\begin{itemize}[<+->]
\item 
	\begin{tabular}{p{3cm}p{5cm}}
{\textbf{Convencional}} & Processamento Numérico \\
{\textbf{Lógica}}  & Processamento Simbólico;
	\end{tabular}
\item 
	\begin{tabular}{p{3cm}p{5cm}}
{\textbf{Convencional}} & Soluções Algorítmicas \\
{\textbf{Lógica}}  & Soluções Heurística;
	\end{tabular}
\item 
	\begin{tabular}{p{3cm}p{5cm}}
{\textbf{Convencional}} & Estruturas de Controle e Conhecimento Integradas  \\
{\textbf{Lógica}}  &  Estruturas de Controle e Conhecimento Separadas.  
	\end{tabular}
\end{itemize}
\end{frame}

\begin{frame}{Porquê Prolog?}
\begin{itemize}[<+->]
\item 
	\begin{tabular}{p{3cm}p{5cm}}
{\textbf{Convencional}} & Difícil Modificação \\
{\textbf{Lógica}}  & Fácil Modificação ;
	\end{tabular}
\item 
	\begin{tabular}{p{3cm}p{5cm}}
{\textbf{Convencional}} & Somente Respostas Totalmente Correctas \\
{\textbf{Lógica}}  & Incluem Respostas Parcialmente Correctas;
	\end{tabular}
\item 
	\begin{tabular}{p{3cm}p{5cm}}
{\textbf{Convencional}} & Somente a Melhor Solução Possível  \\
{\textbf{Lógica}}  &  Incluem Todas as Soluções Possíveis.  
	\end{tabular}
\end{itemize}

\end{frame}

\begin{frame}{Arquitectura Prolog}
\pgfdeclareimage[width=.8\textwidth]{bc}{img/bc}
\begin{figure}
\centering
\pgfuseimage{bc}
\caption{Arquitectura Prolog}
\end{figure}
\end{frame}

\begin{frame}{Nota}
\begin{itemize}
	\item Basicamente um programa Prolog é um conjunto de \alert{Factos} e \alert{Regras};
	\item A interacção é feita através de \alert{Queries}.
\end{itemize}
\end{frame}

\begin{frame}{Exemplo}
	\pgfdeclareimage[height=3.5cm]{familia}{img/familia}
	\begin{example}[Árvore Genealógica]				
		Pretende-se escrever em prolog a árvore genealógica seguinte,
		e representar as relações familiares entre os indivíduos.
		\begin{figure}
			\centering
			\pgfuseimage{familia}
		\end{figure}		
	\end{example}
\end{frame}

\frame[label=showcode]{
\pgfdeclareimage[height=7.5cm]{showcode}{img/showcode}
\begin{figure}
\centering
\pgfuseimage{showcode}
\end{figure}}

\begin{frame}{Programa Heurístico}
	\begin{example}[Alice na Floresta do esquecimento]
		A Alice tinha má memória.
		Um dia entrou na floresta do Esquecimento e esquesceu-se do dia-da-semana.
		Os seus amigos \alert{Coelho} e \alert{Cuco} são visitantes frequentes da floresta. Estes dois
		são criaturas estranhas.\\
		O coelho \alert{mente} às \alert{Segundas}, \alert{Terças} e \alert{Quartas} e diz a \alert{verdade} no \alert{resto da Semana}.
		Por outro lado, o Cuco \alert{mente} às \alert{Quintas}, \alert{Sextas} e \alert{Sábados} \alert{dizendo a verdade no resto dos dias}.
	\end{example}
\end{frame}

\begin{frame}{Programa Heurístico}
	\begin{example}[Alice na Floresta do esquecimento]
		Um certo dia a Alice encontrou estes dois debaixo de uma árvore. Eles fizeram as seguintes declarações:
		\begin{itemize}
			\item Coelho: \alert{Ontem} foi um dos dias que eu \alert{menti};
			\item Cuco: \alert{Ontem} foi um dos dias que eu \alert{menti}.
		\end{itemize}
		
		A Alice foi capaz, usando estas declarações, de deduzir o dia-da-semana em que se encontrava.
	\end{example}
\end{frame}

\begin{frame}{Programa Heurístico}
	\begin{example}[Alice na Floresta do esquecimento]
		Com este exemplo pretende-se: 
		\begin{enumerate}
			\item Escrever uma Base de Conhecimento que descreva esta história;
			\item Escrever um predicado diadehoje/1 que lhe permita saber qual o dia-da-semana.
		\end{enumerate}
	\end{example}
\end{frame}

\againframe{showcode}

\begin{frame}{Extensões Bousi-Prolog}
	\begin{example}[Exemplos Bousi-Prolog]
		Em seguida serão apresentados os seguintes exemplos em Bousi-Prolog
		\begin{itemize}
			\item Programa de cálculo de idades;
			\item Programa de que emula um sistema de ``Information Retrieval'';
			\item Programa de escolha de apartamento inteligente.
		\end{itemize}
	\end{example}
\end{frame}

\againframe{showcode}

\section{E agora?}

\begin{frame}{O que pode ser feito?}
	\begin{itemize}[<+->]
		\item Estudar o sistema que esta a ser desenvolvido na universidade
		de Málaga \alert{FSQL} (Fuzzy SQL) e tentar integrar nos sistemas actuais
		(Neste site propões-se a interacção com SQL Server e Oracle).
		\item Estudar as funções Fuzzy disponibilizadas nos SGBDs actuais 
		(exemplo Soundex e Difference no SQL SERVER).
		\item Estudar os algoritmos de procura de pares em Fuzzy (Busca de informação repetida).
	\end{itemize}
\end{frame}

\begin{frame}{O que há para fazer ainda em Fuzzy?}
	\begin{itemize}[<+->]
		\item Implementar sistemas fuzzy com relações n-árias;
		\item Estudar a possibilidade de desenvolver sistemas com Conjuntos Analógicos;
		\item Modelação de um sistema deductivo de pesquisa de documentação inteligente.
	\end{itemize}
\end{frame}


\section{Sumário}

\begin{frame}{Sumário}
\begin{block}{Recapitulando\ldots}
	\begin{itemize}[<+->]
		\item Vimos motivação do estudo da lógica Fuzzy, bem como
		algumas vantagens da implementação desta;
		\item Estudamos os conceitos Fundamentais da Lógica Fuzzy;
		\item Resolvemos um exercício de Decisão recorrendo 
		à metodologia Mamdani;
		\item Apresentamos o sistema Bousi-Prolog, analizando para
		isso vários exemplos de programas em Prolog e Bousi-Prolog;
		\item Discutimos onde poderia ser usado o conceito na
		nossa corporação e melhorias que poderiam ser feitas a este. 
	\end{itemize}
\end{block}
\end{frame}

\begin{frame}{Dúvidas?}
\pgfdeclareimage[height=4.5cm]{duvidas}{img/duvidas}
		\begin{figure}
			\centering
			\pgfuseimage{duvidas}
		\end{figure}		
\end{frame}

%%%%%%%%%%%%%%%%%%%%%%%%%%%%%%%%%%%%%%%%%%%%%%%%%%%%%%%%%%%%%%%%%%%%%%%%%%%%%

%%%%%%%%%%%%%%%%%%%%%%%%%%%%%%%BIBLIOGRAFIA%%%%%%%%%%%%%%%%%%%%%%%%%%%%%%%%%%%%%%%%%%%%%%
\section*{Referências}

\begin{frame}[allowframebreaks]
  \frametitle<presentation>{Para ler depois}
    
  \begin{thebibliography}{10}
    
  \beamertemplatebookbibitems
  \bibitem{Tanaka1996}
    Kazuo Tanaka.
    \newblock {\em An Introduction to Fuzzy Logic for Practical Applications}.
    \newblock Springer, 1996.

  \bibitem{Shapiro1986}
    Shapiro.
    \newblock {\em The Art of Prolog}.
    \newblock MIT Press, 1986.
 
  \beamertemplatearticlebibitems
  % Followed by interesting articles. Keep the list short. 

  \bibitem{Zadeh1965}
    Zadeh.
    \newblock Fuzzy Sets.
    \newblock {\em Information and Control}, 8(3):338--353,
    1965.
  \end{thebibliography}
\end{frame}

\appendix
\section{\appendixname}
\frame{\tableofcontents}
\subsection{Material Adicional}

\frame[label=eqpertencadetail]{
  \frametitle{Definições Fuzzy --- Etiquetas e Conjuntos Fuzzy Detalhe}

$$
		\mu_X(x_0) = \left \{ \begin{array}{ll}
				$1$ & \textrm{sse $x \in \mathcal{C}$} \\
				$0$ & \textrm{sse $x \notin \mathcal{C}$} \\
				$0$ \le \mu_X(x_0) \le 1 & \textrm{sse $x \sim \in \mathcal{C}$} \\
			\end{array} \right . 
		$$	

    \hfill\hyperlink{eqpertenca}{\beamerreturnbutton{Voltar}}
}

\end{document}


